\section{Python Abstract Syntax and Parser}\label{sec:pysyn}
\subsection{Python Abstract Syntax}

We first define the abstract syntax for the Python programming language.
The syntax of Python is described in the Python Language Reference \cite{pythonref}.
The reference provides a full grammar specification based on the extended PEG
and a detailed explanation of syntactic components in each section.
We manually examined the grammar and the details
to define the Python abstract syntax.
The syntax is composed of three syntactic components: expressions, statements,
and top-level components.

\begin{figure}[!ht]
\begin{tabular}{lrll}
  \nmodule & := & \mul{\nstmt}  \\
  \nstmt & ::= & \kdef ~ \nid ~ \sparen{\nargs} & \desc{FunDef} \\ 
  & $|$ & \kclass ~ \nid ~ \sparen{\mul{\nexpr} \mul{\nkeyword}} ~ \kcolon ~ \mul{\nstmt} & \desc{ClassDef} \\
  & $|$ & \nexpr ~ \oassign ~ \nexpr & \desc{Assign} \\
  & $|$ & \kfor ~ \nexpr ~ \kin ~ \nexpr ~ \kcolon ~ \mul{\nstmt} & \desc{ForLoop} \\
  & $|$ & \kwhile ~ \sparen{\nexpr} ~ \kcolon ~ \mul{\nstmt} & \desc{WhileLoop} \\
  & $|$ & \kif ~ \sparen{\nexpr} ~ \kcolon ~ \mul{\nstmt} ~ \op{(\kelse ~ \kcolon ~ \mul{\nstmt})}& \desc{If} \\
  & $|$ & \kwith ~ \mul{\nwithitem} ~ \kcolon ~ \mul{\nstmt} & \desc{With} \\
  & $|$ & \kimport ~ \mul{\nalias} & \desc{Import} \\
  & $|$ & \kfrom ~ \nint ~ \op{\nid} \kimport ~ \mul{\nalias} & \desc{ImportFrom} \\
  & $|$ & \nexpr & \desc{ExprStmt} \\

  \nexpr & ::= & \nexpr ~ \nboolop ~ \nexpr & \desc{BoolOp} \\
  & $|$ & \nexpr ~ \nbinop ~ \nexpr & \desc{BinaryOp} \\ 
  & $|$ & \nunop ~ \nexpr & \desc{UnaryOp} \\ 
  & $|$ & \lparen{\mul{\nexpr}} & \desc{List} \\ 
  & $|$ & \sparen{\mul{\nexpr}} & \desc{Tuple} \\ 
  & $|$ & \nexpr ~ \sparen{\mul{\nexpr} \mul{\nkeyword}} & \desc{Call} \\
  & $|$ & \nconstant & \desc{Constant} \\
  & $|$ & \nexpr {\tt .}\nid& \desc{Attribute} \\
  & $|$ & \nid & \desc{Name} \\

  \nboolop & ::= & \oand ~ $|$ ~ \oor & \desc{BoolOperator} \\
  \nbinop & ::= & \oand ~ $|$ ~ \osub ~ $|$ ~ \omul & \desc{BinOperator} \\
  \nunop& ::= & \kinvert ~ $|$ ~ \knot ~ $|$ ~ \oadd ~ $|$ ~ \osub & \desc{UnOperator} \\
  \ncompop& ::= & \oeq ~ $|$ ~ \oneq ~ $|$ ~ \olt ~ $|$ ~ \olte ~ $|$ ~ \ogt ~ $|$ ~ \ogte ~ $|$ ~ \ois ~ $|$ ~ \onis ~ $|$ ~ \oin ~ $|$ ~ \onin & \desc{CompOperator}\\
  \nargs & ::= & \mul{(\narg ~ \op{(\oassign ~ \nexpr)})}, ~ \mul{(\narg ~ \op{(\oassign ~ \nexpr)})}, ~ \op{\narg}, ~ \mul{(\narg ~ \op{(\oassign ~ \nexpr)})}, ~ \op{\narg} & \desc{Arguments}\\
  \narg & ::= & \nid ~ \op{\nexpr}~\op{\nstr} & \desc{Argument} \\
  \nkeyword & ::= & \op{\nid} \oassign \nexpr & \desc{Keyword} \\ 
  \nalias & ::= & \nid ~\mul{(.\nid)} \op{(\kas ~ \nid)} & \desc{Alias} \\
  \nwithitem & ::= & \nexpr ~ \op{(\kas ~ \nexpr)} & \desc{WithItem}\\

  \nconstant & ::= & \knone & \desc{NoneLiteral} \\
  & $|$ & \nint & \desc{IntLiteral} \\
  & $|$ & \nfloat & \desc{FloatLiteral} \\
  & $|$ & \ncomplex & \desc{ComplexLiteral} \\
  & $|$ & \nstr & \desc{StringLiteral} \\
  & $|$ & \nbool & \desc{BooleanLiteral} \\
  & $|$ & \sparen{\mul{\nconstant}} & \desc{TupleLiteral} \\
  & $|$ & {\tt ...} & \desc{Ellipsis} \\
  \ntypignore & ::=  & \mul{\nint} & \desc{TypeIgnore} \\
  \nid & $\in$ & \did \\
  \nstr & $\in$ & \dstr \\
  \nbool & $\in$ & \{{\tt True}, {\tt False}\}\\
  \nint & $\in$ & $\mathbb{Z}$ \\
  \nfloat & $\in$ & $\mathbb{R}$ \\
  \ncomplex & $\in$ & $\mathbb{C}$ \\
\end{tabular}
  \caption{Python abstract syntax}
  \label{fig:parse:abssyntax}
\end{figure}

Expressions are parts of the code that evaluates a value.
Python has 5 kinds of primitive values, 
which are numbers, strings, booleans, the value {\itshape None}, and {\itshape Ellipsis}.
The value None is used to denote an undefined value,
similar to the value {\itshape null} in Java.
The value Ellipsis corresponds to the notation ``...",
which can be used as a special placeholder meaning expansion of the sequence. 
Python has composite types of a tuple, list, set, map, and custom classes.
The expression syntax defines ways to build up values
and complex expressions such as operators, comprehension, and function calls. 

Statements are parts of the code that changes the program state,
such as variable binding or control flow. 
Python statements are categorized into 
simple statements which denote a single state-changing step, 
or compound statements that are composed of multiple statements.

Simple statements include assignment statements and import statements.
Assignment statements are used to declare a new variable and its value
or update a variable value.
Import statements are used to specify a module, load it
and get definitions of the module into the current namespace.
In import statements, each target module and its aliased name is represented
as an alias.   
Note that the procedure call is a special case of expression statements,
where the expression is a function call.

Compound statements include conditional statements, loop statements,
and definitions for functions and classes.
Additionally, the with statement is a special kind of compound statement.
With statement is identical to the assignment statement 
in a way that it binds an expression to a name, 
but the statement additionally adds implicit calls
to the object's methods related to initialization and destruction.
Similar to the alias in an {\tt import} statement,
with statements use WithItem to represent then name and expression.

Top-level components are the representation of the program
in the different execution environments of the Python interpreter.
For example, a module represents a Python code file, composed of
multiple definitions specified by statements. 
We summarize the Python abstract syntax in figure \ref{fig:parse:abssyntax}.
The full Python abstract syntax is attached in the supplementary material.

\subsection{Parser Implementation}

Section 10 of the Python language reference\cite{pythonref}
provides a full Python grammar specification.
The specification is described in the mixture of 
extended BNF and PEG\cite{bryan2004peg}. We implemented the Python parser
according to the grammar specification with the Scala parser combinators 
library\cite{scalaparser}. We utilized the library's implementation of
the Packrat parser\cite{bryan02packrat}, which is a time-efficient parser
for the PEG.

