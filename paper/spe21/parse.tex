\section{Python Abstract Syntax and Parser}\label{sec:pysyn}
\subsection{Python Abstract Syntax}

We first formalize the abstract syntax of the Python programming language.
The Python syntax is described in the 
Python Language Reference \cite{pythonref}.
The reference provides a full grammar specification based on the extended PEG
and a detailed explanation of syntactic components in each section.
We manually examined the grammar and the details
to define the Python abstract syntax.
The figure \ref{fig:parse:abssyntax} illustrates the simplified
version of the Python abstract syntax.
The Python languague is composed of three major components:
modules, statements, and expressions.

The top-level component of Python is a module, which is
a sequence of statements.
The module represents a program source code file,
containing multiple definitions of functions or classes.

Statements are parts of the code that changes the program state.
For instance, assignment statement chanages the
value stored in the variable, and if-else statement changes
the control flow of the program.
In addition to the basic imperative statements,

Expressions are parts of the code that evaluates a value.
Python has composite types of a tuple, list, set, map, and custom classes.
The expression syntax defines ways to build up values
and complex expressions such as operators, comprehension, and function calls. 

\begin{figure}[!ht]
\begin{tabular}{lrll}
  \nmodule & := & \mul{\nstmt}  \\
  \nstmt & ::= & \kdef ~ \nid ~ \sparen{\nargs} ~ \kcolon ~ \mul{\nstmt} & \desc{FunDef} \\ 
  & $|$ & \kclass ~ \nid ~ \sparen{\mul{\nexpr} \mul{\nkeyword}} ~ \kcolon ~ \mul{\nstmt} & \desc{ClassDef} \\
  & $|$ & \nexpr ~ \oassign \nexpr & \desc{Assign} \\
  & $|$ & \kfor ~ \nexpr ~ \kin ~ \nexpr ~ \kcolon ~ \mul{\nstmt} & \desc{ForLoop} \\
  & $|$ & \kwhile ~ \sparen{\nexpr} ~ \kcolon ~ \mul{\nstmt} & \desc{WhileLoop} \\
  & $|$ & \kif ~ \sparen{\nexpr} ~ \kcolon ~ \mul{\nstmt} ~ \op{(\kelse ~ \kcolon ~ \mul{\nstmt})}& \desc{If} \\
  & $|$ & \kwith ~ \mul{\nwithitem} ~ \kcolon ~ \mul{\nstmt} & \desc{With} \\
  & $|$ & \kimport ~ \mul{\nalias} & \desc{Import} \\
  & $|$ & \kfrom ~ \nint ~ \op{\nid} \kimport ~ \mul{\nalias} & \desc{ImportFrom} \\
  & $|$ & \nexpr & \desc{ExprStmt} \\

  \nexpr & ::= & \nexpr ~ \nboolop ~ \nexpr & \desc{BoolOp} \\
  & $|$ & \nexpr ~ \nbinop ~ \nexpr & \desc{BinaryOp} \\ 
  & $|$ & \nunop ~ \nexpr & \desc{UnaryOp} \\ 
  & $|$ & \lparen{\mul{\nexpr}} & \desc{List} \\ 
  & $|$ & \sparen{\mul{\nexpr}} & \desc{Tuple} \\ 
  & $|$ & \nexpr ~ \sparen{\mul{\nexpr} \mul{\nkeyword}} & \desc{Call} \\
  & $|$ & \nconstant & \desc{Constant} \\
  & $|$ & \nexpr {\tt .}\nid& \desc{Attribute} \\
  & $|$ & \nid & \desc{Name} \\

  \nboolop & ::= & \oand ~ $|$ ~ \oor & \desc{BoolOperator} \\
  \nbinop & ::= & \oand ~ $|$ ~ \osub ~ $|$ ~ \omul & \desc{BinOperator} \\
  \nunop& ::= & \kinvert ~ $|$ ~ \knot ~ $|$ ~ \oadd ~ $|$ ~ \osub & \desc{UnOperator} \\
  \nargs & ::= & \mul{(\narg ~ \op{(\oassign ~ \nexpr)})}, ~ \mul{(\narg ~ \op{(\oassign ~ \nexpr)})}, ~ \op{\narg}, ~ \mul{(\narg ~ \op{(\oassign ~ \nexpr)})}, ~ \op{\narg} & \desc{Arguments}\\
  \narg & ::= & \nid ~ \op{\nexpr}~\op{\nstr} & \desc{Argument} \\
  \nkeyword & ::= & \op{\nid} \oassign \nexpr & \desc{Keyword} \\ 
  \nalias & ::= & \nid ~\mul{(.\nid)} \op{(\kas ~ \nid)} & \desc{Alias} \\
  \nwithitem & ::= & \nexpr ~ \op{(\kas ~ \nexpr)} & \desc{WithItem}\\

  \nconstant & ::= & \knone & \desc{NoneLiteral} \\
  & $|$ & \nint & \desc{IntLiteral} \\
  & $|$ & \nfloat & \desc{FloatLiteral} \\
  & $|$ & \nstr & \desc{StringLiteral} \\
  & $|$ & \nbool & \desc{BooleanLiteral} \\
  & $|$ & \sparen{\mul{\nconstant}} & \desc{TupleLiteral} \\
  \nid & $\in$ & \did \\
  \nstr & $\in$ & \dstr \\
  \nbool & $\in$ & \{{\tt True}, {\tt False}\}\\
  \nint & $\in$ & $\mathbb{Z}$ \\
  \nfloat & $\in$ & $\mathbb{R}$ \\
\end{tabular}
  \caption{Python abstract syntax}
  \label{fig:parse:abssyntax}
\end{figure}


