\section{Background}\label{sec:background}
\subsection{TensorFlow ML Models}

TensorFlow\cite{tensorflow} is a machine learning platform library
developed in 2016 by Google Brains.
The library is based on a dataflow graph.
The dataflow graph is a abstraction of the computations
composing the model.
In dataflow graphs, nodes represent operations and
edges represent flow of tensors between the operations.
One important innovation of TensorFlow compared to its predecessors
is that the mutable states are also represented as oerations
within the dataflow graph. 
A variable node maintains a mutable buffer that can be read or modified.
This approach allows TensorFlow to define the model parameters
and updating mechanism at the same time.
TensorFlow also allows users to easily distribute model computation over
such as GPUs or TPUs.
Each operation on the dataflow graph defines kernel,
a low-level implementation of the computation.
Because the dataflow graph explicitly specifies how
and where the subcomputation results should be sent and recevied,
users can easliy assign each kernels to different devices and construct
the whole computaion from individual kernels and the dataflow graph.

\lstinputlisting[language=Python, caption=Example Tensorflow1 model code]
{tensorflow1_mnist.py}

We explain a typical form of TensorFlow ML model codes by an example.
The figure illustrates a simple neural network with 2 hidden layer.
The code starts with importing the TensorFlow module.
Typically, TensorFlow ML training codes is divided into two stages.
In first stage, the model and training algorithm is defined.
The model is defined by using API functions to construct the dataflow graph.
Then loss function is defined, usually using the pre-defined functions.
Optimization is also defined by using API functions to construct
an Optimizer object, then assigning the target function.
Actual training is done in the second stage of the code.
In the second stage, the TensorFlow runtime repeats the training step
as each data from the training dataset is fetched and
the model is optimized against it.

(how TF version 2 differs : by example codes)

\subsection{Distributed Training and Horovod Framework}

(introduce general concept of distributed training)

Distributed training is a technique to boast efficiency in ML training
by parallelizing computation workloads in time.
Training an ML model is an expensive process that involves repetitive
arithmetic computation over multiple tensors,
which can be accelerated by multicore devices such as GPU or TPU. 
Distribtue training takes advantage of multiple accelerator devices to
The total training workload is divided into smaller, independent workloads
so they can be assigned into multiple accelerator devices and
processed simultaneously. 
In recent years, several works proposed application of distributed
deep learning in various fields.

(introduce model-parallel and data-parallel approach\cite{approaches2019Mao})

There are two approaches in distributed ML training.
In model-parallel approach, a large ML model is divided into separate unit
of computation pipes and assigned to different accelerators. The whole model
is pipelined over series of accelerators and so the forward-propagation and
back-propagation can be done in the distributed system.
In data-parallel apporach, however, multiple instances of same ML model
is assigned to each accelerators. The total training dataset is instead
divided into a number of smaller datasets which then assigned to different
accelerators. The training proceeds by simultaneously calculating gradients
in each accelerators. Then the gradients are averaged before being 
back-propagated into the model parameters.

(introduce facebook's data-parallel approach and improvements)

Facebook's experiment \cite{facebook2018} provides empirical evidence of
scalability of data-parallel distributed training. In the paper, 
halving/doubling-based allreduce algorithm is used to reduce communication cost 
of averaging gradients between the GPUs. 
Allreduce algorithm eliminates performance bottleneck of traditional
averaging algorithm by aggregating the average value through series of
point-to-point communication between GPUs. 
Experimental results of the paper show that the performance linearly
scales from 8-GPUs to 352 GPUs, while achieving over 90\% of ideal performance.

(introduce horovod)

Inspired by Facebook's results, Uber Engineering developed a distributed DL
training framework namely Horovod \cite{sergeev2018horovod}. It is published in
2017 as a part of DL toolkit for Michelangelo, a ML-as-a-service platform.
The main motivation of Horovod is to scale single-GPU DL model training
into distributed model training. Similar to the Facebook's approach,
Horovod implements allreduce algorithm in NCCL. NCCL is collective communication
library from NVIDIA, which provides various performant algorithms including
allreduce. In addition, Horovod introduces high-level API that enables
convenient use of the distributed training.  

(horovod provides convenient API to change existing model into distributed one)

(example of using horovod to distribute model
