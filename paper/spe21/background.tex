\section{Background}\label{sec:background}
\subsection{TensorFlow Deep Learning Models}


% Plan
% explain typical form of tf model by code examples
% will explain two categories
% 1. tf 1.x version models
% 2. tf 2.x version models

TensorFlow\cite{tensorflow} is a machine learning platform library
developed by Google Brains.
ML developers can use TensorFlow-provided APIs to 
methods given by TensorFlow library.
This section describes two representative forms of TensorFlow
model codes, which are TensorFlow 1.x version format and 
TensorFlow 2.x version format.

\begin{figure}[ht!]
\lstinputlisting[language=Python]
{tensorflow1_mnist.py}
  \caption{TensorFlow 1.x model example}
\label{fig:back:tf1}
\end{figure}

% 위 코드는 tf 1.x 버전 모델의 코드이다.
% 1.x 버전의 코드에서는 모델의 구조를 Tensorflow API를 통해 정의한다.
% 예를 들어 위 코드는 히든 레이어가 두개인 모델을 4~19번 줄에서 정의하고 있다.
% 8-9번줄은 첫 번째 히든 레이어를,
% 13-14번줄은 두 번째 히든 레이어를 정의하는데,
% 이 때 각 히든 레이어의 weight와 bias를 정의하기 위해
% tf.Variable이라는 API를 사용하고 있음을 알 수 있다.
% 18-19번줄은 모델의 loss 계산 결과값과 optimization algorithm을 정의하고 있다.
% 1.x 버전에서 모델을 훈련시키기 위해서는 Session API를 사용해서
% 훈련 루프를 직접 작성한다.
Figure~\ref{fig:back:tf1} illustrates an example of TensorFlow 1.x model.
In TensorFlow 1.x version, developers use TensorFlow APIs to define model
structures.
In figure~\ref{fig:back:tf1}, for instance, the lines 4 to 15
defines a neural network model with two hidden layers.
 
\begin{figure}[ht!]
\lstinputlisting[language=Python]
{tensorflow2_ex.py}
  \caption{TensorFlow 2.x model example}
\label{fig:back:tf2}
\end{figure}

% 위 코드는 tf 2.x 버전의 코드이다.
% 2.x 버전에서는 모델 구조를 6~15줄과 같이 keras API를 사용하여
% 간편하게 정의할 수 있다.
% 예를 들어 위 코드에서는 keras의 Sequential API를 이용하여
% 두 개의 convolutional 레이어와 두 개의 fully connected 레이어를
% 순서대로 연결한 모델을 정의했다.
% 이 모델을 훈련시키기 위해서 tf 2.x 모델에서는 GradientTape API를 사용한다.
% 위 코드의 22번줄과 같이 with문을 이용하여 GradientTape 오브젝트를
% 생성하면, 해당 with 문 내에서 실행되는 계산들에 대한 gradient를
% 자동으로 계산할 수 있다.
% 이렇게 자동으로 계산된 gradient를 26-27번줄에서 사용하여
% 모델 파라미터를 gradient descent를 통해 최적화한다.

Figure~\ref{fig:back:tf2} illustrates a TensorFlow 2.x model example.
The main difference of Tensorflow 2.x with TensorFlow 1.x is eager evaluation.
In TensorFlow 2.x codes, developers can 
Explicit calls to TensorFlow APIs are replaced by Python function calls,
and TensorFlow variables are replaced by function parameters.
This way, developers can easily define model components and compose them. 
In addition, TensorFlow 2.0 integrates Keras\cite{keras},
a deep learning library that provides high-level abstractions for layers.

\subsection{Horovod Distributed Training Framework}

\begin{figure}
 \lstinputlisting[language=Python]
{horovod_ex.py}
  \caption{Horovod distributed model example}
\label{fig:back:hvd} 
\end{figure}

% Horovod는 기존 ml 모델을 분산 훈련 모델으로 변환하기 위한 라이브러리이다.
% Horovod는 model-parallel 방식으로 ml 모델을 분산 훈련을 위한 모델로 바꾼다.
% 
Inspired by Facebook's results, Uber Engineering developed a distributed DL
training framework Horovod \cite{sergeev2018horovod}. 
The main motivation of Horovod is to scale single-GPU DL model training
into distributed model training using data-parallel approach. 
Similar to Facebook's approach, Horovod implements an allreduce algorithm. 
In addition, Horovod introduces a high-level API that enables
the convenient use of distributed training.  
