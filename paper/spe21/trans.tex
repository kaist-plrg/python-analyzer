\section{Code Transformation}\label{sec:trans}
\subsection{Transformation Rule for Distributed TensorFlow ML model}

In this section, we describe formalization of the transformation rule
for distributed DL traininig with set of code examples and formal notions
related to them.
There are two main points in design of the formalization.
First, we informally understand the code transformation rule as methods to
1) select the parts of codes to modify and 2) construct new code by reusing
parts of the input code.
To automate the transformation process, we need code transformation in a form 
that is implements above actions as a program.
Second, as mentioned earlier, to correctly transform tranining codes with different
API patterns, we must apply different transformation rules.
The formalization should also describe different transformation rules
for different API patterns.
In this end, we define a transformation rule for each API pattern as 
a set of pure function that takes ASTs as input and returns ASTs as output.
We call the function as \textit{transform function}.
Each transform function is defined to get an AST of given type and return an AST
of the same type. For instance, a transform function for statement,
$\fkstmt$, takes a statement AST and return a statement AST.

\begin{figure}[ht!]
  \centering
  \begin{subfigure}[t]{0.48\textwidth}
    \begin{lstlisting}[language=Python]
import tensorflow as tf

dataset = ...
model = ...
optim = tf.optimizers.Adam(0.001) 

for (x, y) in dataset.take(10000):
  with tf.GradientTape() as tape:
    pred = model(x)
    loss_value = loss(y, pred) 

  gradients = tape.gradient(loss_value, model.trainable_variables)
  optim.apply_gradient(zip(gradients, model.trainable_variable)\end{lstlisting}
    \caption{Original DL training code}
  \end{subfigure}
  \hspace{5mm}
  \begin{subfigure}[t]{0.48\textwidth}
    \begin{lstlisting}[language=Python]
import tensorflow as tf
import horovod.tensorflow as hvd

dataset = ...
model = ...
optim = tf.optimizers.Adam(0.001 * hvd.size()) 

for (x, y) in dataset.take(10000):
  with tf.GradientTape() as tape:
    pred = model(x)
    loss_value = loss(y, pred) 
  tape = hvd.DistributedGradientTape(tape)

  gradients = tape.gradient(loss_value, model.trainable_variables)
  optim.apply_gradient(zip(gradients, model.trainable_variable)\end{lstlisting}
    \caption{Distributed DL training code}
  \end{subfigure}
  \caption{Code examples}
  \label{fig:trans:ex}
\end{figure}

Figure \ref{fig:trans:ex} shows an example of the input single-GPU code and
the coresponding transformed output distributed code. 
Some unnecessary details of the codes are removed from the codes.
There are three transformed parts in the distributed training code.
First, a new import statement that imports {\tt horovod.tensorflow} modules
is added in line 2. 
The statement is added right after the original {\tt tensorflow} module
import statement.
Second, in line 6, an argument of the constructor function 
{\tt tf.optimizers.Adam} is modified. The newly changed argument multiplies
{\tt hvd.size()} onto the original argument.
Third, in line 12, a new statement that assigns the variable {\tt tape}
is added. The assignment wraps the original {\tt GradientTape} object with
the Horovod library function, {\tt hvd.DistributedGradientTape}.
Except the above mentioned parts, the original code is not modified.     

We formalize the transformation rule as set of pure functions
that take ASTs as input and return ASTs as output.
For each training API pattern, we define  

\subsection{Description of Transform functions}

As mentioned earlier, training codes of different API patterns
require different transformation rules to correctly transform them.
This section explains formal transform functions for each API patterns.

\subsubsection{Transform function for Session pattern}

To correctly transform Session pattern trainig codes into corresponding
distributed training codes, the {\tt Optimizer} instance must be modified.
Figure \ref{fig:trans:sessiontrans} shows an pair of code examples
illustrating the required transformation for Session pattern case.
In line 1 of figure \ref{fig:trans:sessiontrans}(b),
the {\tt learning\_rate} argument is scaled by {\tt hvd.size()}.
Then the line 2 is added, which wraps the {\tt Optimizer} instance
by designated Horovod API.

\begin{figure}[ht!]
  \begin{subfigure}[t]{0.45\textwidth}
    \begin{lstlisting}[language=Python]
optimizer = tf.train.MomentumOptimizer(learning_rate = 0.01)

with tf.Session() as sess:
    for step in range(num_epochs): 
        sess.run(optimizer, feed_dict)
    \end{lstlisting}
    \caption{Original training code of Session pattern}
  \end{subfigure}
  \hspace{5mm}
  \begin{subfigure}[t]{0.45\textwidth}
    \begin{lstlisting}[language=Python]
optimizer = tf.train.MomentumOptimizer(learning_rate = 0.01 * hvd.size())
optimizer = hvd.DistributedOptimizer(optimizer)

with tf.Session() as sess:
    for step in range(num_epochs): 
        sess.run(optimizer, feed_dict)
    \end{lstlisting}
    \caption{Distributed training code of Session pattern}
  \end{subfigure}
  \caption{Code example of Session pattern training code}
  \label{fig:trans:sessiontrans}
\end{figure}

The above transformation is formalized as shown in the figure 
\ref{fig:trans:sessrule}. The transform function stated in the figure
matches statements that assigns a evaluation result of function call expression
on the right-hand side.
If an assign statement is matched, the pattern guard checks if the
callee function is the {\tt Optimizer} instance constructor function.
Note that the pattern guard checks this by subclass relation between
the function expression and {\tt tensorflow.keras.optimizers.Optimizer}.
This way, the pattern guard can accept class constructors for {\tt Optimizer}
instances, including both TensorFlow library and user-defined classes.

The transform function proceeds to return a sequence of two statements;
the first one is modified version of the input assign statement,
and the second one is additional assign statement for wrapping the optimizer instance.
The first statement modifies the original {\tt learning\_rate} argument
by constructing a new expression, {\tt \nexprsubs{2i} * hvd.size()}.
The transform function can easily construct new ASTs by using
the pattern variables so that the part of the input statement can be conveniently
referred. This is also true for building the second output statement, which
reused the {\tt \nidsubs{r}} to construct a new assign statment.

\begin{figure}[h]
\noindent
\begin{longtable}{l}
  \tstmt{\nidsubs{r} \oassign \nexprsubs{1} \sparen{\nexprsubs{11} ... \nexprsubs{1n} ~ \op{(\nidsubs{1} \oassign)} \nexprsubs{21} ... \op{(\nidsubs{k} \oassign)} \nexprsubs{2k}} }{\smodenv} = \\
\inden \ktif ~ \nexprsubs{1} \ktsubty ~ {\tt tensorflow.keras.optimizers.Optimizer} ~ \ktthen\\
  \inden\inden \ktif ~ \nidsubs{i} ~ \kteq ~ {\tt learning\_rate} ~ \ktwhen ~ 1 $\leq$ i $\leq$ k ~ \ktthen\\
  \inden\inden\inden ([\nidsubs{r} \oassign \nexprsubs{1} \sparen{\nexprsubs{11} ... \nexprsubs{1n} ~ \op{(\nidsubs{1} \oassign)} \nexprsubs{21} ... \nidsubs{i} \oassign \nexprsubs{2i} {\tt * hvd.size()}\\
  \inden\inden\inden\inden ... \op{(\nidsubs{k} \oassign)} \nexprsubs{2k}} \optypcomm \\
  \inden\inden\inden {\tt \nidsubs{r} = hvd.DistributedOptimizer(\nidsubs{r})}],
  \smodenv[\optmizer $\mapsto$ \nidsubs{r}])\\
\end{longtable}
  \caption{Session pattern transform function: Optimizer learning rate scaling and wrapping}
  \label{fig:trans:sessrule}
\end{figure}

\subsubsection{Transform function for MonitoredSession pattern}

The MonitoredSession pattern codes use {\tt MonitoredSession} instances
to initiate and automatically repeat the tranining process.
To correctly transform MonitoredSession pattern training codes into the
distributed training codes,

\begin{figure}[ht!]
  \begin{subfigure}[t]{0.45\textwidth}
    \begin{lstlisting}[language=Python]
with tf.train.MonitoredTrainingSession(hooks=hooks) as mon_sess:
    while not mon_sess.should_stop():
        mon_sess.run()
    \end{lstlisting}
    \caption{Original training code of MonitoredSession pattern}
  \end{subfigure}
  \hspace{5mm}
  \begin{subfigure}[t]{0.45\textwidth}
    \begin{lstlisting}[language=Python]
with tf.train.MonitoredTrainingSession(hooks=hooks.append(hvd.BroadcastGlobalVariablesHook(0)) as mon_sess:
    while not mon_sess.should_stop():
        mon_sess.run()
    \end{lstlisting}
    \caption{Distributed training code of MonitoredSession pattern}
  \end{subfigure}
  \caption{Code example of MonitoredSession pattern training code}
  \label{fig:trans:monsesstrans}
\end{figure}

The codes for MonitoreSession pattern additionally need to modify the
{\tt StopAtStepHook} instance in order to scale down the number of
training steps. The {\tt Hook} instances are given as an list argument of
the {\tt MonitoredSesion} instance constructor.
To modify the number of traning steps, the transform function
identifies by expression that constructs the {\tt StopAtStepHook} instance
and transform its {\tt last\_step} keyword argument.

\begin{figure}[h]
 \noindent
\begin{tabular}{l}
  \texpr{\nexprsubs{1} \sparen{\nexprsubs{11} ... \nexprsubs{1n} ~ \op{(\nidsubs{1} \oassign)} \nexprsubs{21} ... \op{(\nidsubs{k} \oassign)} \nexprsubs{2k}}}{\smodenv} = \\
  \inden \ktif ~ \smodenv(\tflowc) ~ \kteq ~ \nidsubs{t} ~ \ktand ~ \nexprsubs{1} ~ \kteq ~ {\tt \nidsubs{t}.train.StopAtStepHook} ~ \ktthen \\
  \inden\inden \ktif ~ \nidsubs{i} ~ \kteq ~ {\tt last\_step} ~ \ktwhen ~ 1 $\leq$ i $\leq$ k ~ \ktthen\\
  \inden\inden\inden \nexprsubs{1} \sparen{\nexprsubs{11} ... \nexprsubs{1n} ~ \op{(\nidsubs{1} \oassign)} \nexprsubs{21} ... \nidsubs{i} \oassign \nexprsubs{2i} {\tt // hvd.size()}\\
  \inden\inden\inden\inden ... \op{(\nidsubs{k} \oassign)} \nexprsubs{2k}}\\
  \inden\inden \ktelse \\
  \inden\inden\inden \nexprsubs{1} \sparen{\nexprsubs{11} {\tt // hvd.size()} ... \nexprsubs{1n} ~ \op{(\nidsubs{1} \oassign)} \nexprsubs{21} ... \op{(\nidsubs{k} \oassign)} \nexprsubs{2k}} \\
  \inden \ktelse ~ \texpr{\nexprsubs{1}}{\smodenv} \sparen{\texpr{\nexprsubs{11}}{\smodenv}  ... \texpr{\nexprsubs{1n}}{\smodenv} \\
  \inden\inden \op{(\nidsubs{1} \oassign)} \texpr{\nexprsubs{21}}{\smodenv} ... \op{(\nidsubs{k} \oassign)} \texpr{\nexprsubs{2k}}{\smodenv}}\\
\end{tabular}\\\vpar
\caption{MonitoredSession pattern transform function: Modifying {\tt StopAtStepHook} instance}
\end{figure}

\subsubsection{Transform function for GradientTape pattern}

In training codes matched in the GradientTape pattern,
the {\tt GradientTape} instance is wrapped with
{\tt hvd.DistributedGradientTape} function.
The {\tt GradientTape} instances are created by {\tt with} statements
in the training codes, which automatically invokes initialization
and destruction functions before and after the {\tt with} statement scope.
The transform function matches the {\tt with} statement that
creates the {\tt GradientTape} instance and inserts the
wrapping statement after it.
Figure \ref{fig:trans:gtape} illustrates the transform function definition
regarding the {\tt GradientTape} wrapping.

\begin{figure}[h]
\noindent
\begin{tabular}{l}
  \tstmt{\optypcomm ~ \kwith ~ \mul{\nwithitem} ~ \kcolon ~ \mul{\nstmt}}{\smodenv} = \\
  \inden \ktlet ~ \mul{\nwithitem}$'$, \smodenvsubs{1} \kteq ~ \twwithitem{\mul{\nwithitem}}{\smodenv} \ktin \\
  \inden \ktlet ~ \mul{\nstmt}$'$, \smodenvsubs{2} \kteq ~ \tsstmt{\mul{\nstmt}}{\smodenvsubs{1}} \ktin \\
  \inden \ktif ~ \smodenvsubs{1} \envsub ~ \smodenv ~ \kteq ~ [\gtape $\mapsto$ \nid] ~ \ktthen\\
  \inden\inden ([\optypcomm ~ \kwith ~ \mul{\nwithitem}$'$ ~ \kcolon ~ \mul{\nstmt}$'$, \\
  \inden\inden \nid ~ \oassign {\tt hvd.DistributedGradientTape(\nid)}], \smodenvsubs{2})\\
  \inden \ktelse ~ ([\optypcomm ~ \kwith ~ \mul{\nwithitem}$'$ ~ \kcolon ~ \mul{\nstmt}$'$], \smodenvsubs{2})
\end{tabular}
  \caption{GradientTape pattern transform function: wrapping {\tt GradientTape} instance}
  \label{fig:trans:gtape}
\end{figure}

In addition, the variable broadcast should be manually done with
{\tt hvd.broadcast\_variables} function.
Because the variable broadcast should be done exactly once at the start of the
training, we utilize a global boolean variable {\tt hvd\_broadcast\_done}
that tracks whether the broadcast is already performed or not.
The variable {\tt hvd\_broadcast\_done} is declared within the Horovod
initialization codes and initialized to {\tt False} as described in
figure \ref{fig:trans:gtape2}.

\begin{figure}[h]
\begin{tabular}{l}
  \tstmt{\kimport ~ \mul{\nalias}}{\smodenv} = \\
  \inden \ktlet ~ \smodenvsubs{1} ~ \kteq ~ \taalias{\mul{\nalias}}{\smodenv} \ktin \\
  \inden \ktif ~ \smodenvsubs{1} ~ \envsub ~ \smodenv ~ \kteq ~ [\tflow $\mapsto$ \nid] ~ \ktthen \\
  \inden\hspace{1em} ([\kimport ~ \mul{\nalias}, \\
  \inden\hspace{1em} \kimport ~ {\tt horovod.tensorflow} \kas ~ {\tt hvd}, \\
  \inden\hspace{1em} {\tt hvd\_broadcast\_done} \oassign {\tt False}, \\
  \inden\hspace{1em} {\tt hvd.init()}, \\
  \inden\hspace{1em} {\tt gpus = \nid.config.experimental.list\_physical\_devices(`GPU')}, \\
  \inden\hspace{1em} {\tt for gpu in gpus: \nid.config.experimental.set\_memory\_growth(gpu, True)},\\
  \inden\hspace{1em} {\tt if gpus: \nid.config.experimental.set\_visible\_devices(gpus[hvd.local\_rank()], `GPU')}], \smodenvsubs{1})\\
  \inden \ktelse~([\kimport ~ \mul{\nalias}], \smodenvsubs{1})
\end{tabular}
  \caption{GradientTape pattern transform function: initializing {\tt hvd\_broadcast\_done}}
  \label{fig:trans:gtape2}
\end{figure}

The statements that perform the variable broadcast are inserted after
the {\tt Optimizer} instance applies the training gradient.
The corresponding transform function is described in figure \ref{fig:trans:gtape3}.
The inserted codes check and modify the boolean value of 
{\tt hvd\_broadcast\_done} to ensure that the variable broadcast
happens only once.

\begin{figure}[h]
\begin{longtable}{l}
  \tstmt{\nexprsubs{1} \sparen{\nexprsubs{11} ... \nexprsubs{1n} ~ \op{(\nidsubs{1} \oassign)} \nexprsubs{21} ... \op{(\nidsubs{k} \oassign)} \nexprsubs{2k}}}{\smodenv} = \\
  % variable broadcasting
  \inden \ktif  ~ \smodenv(\optmizer) ~ \kteq ~ \nidsubs{t} ~ \ktand ~ \nexprsubs{1} ~ \kteq ~ {\tt \nidsubs{t}.apply\_gradients} ~ \ktthen\\
  \inden\inden \ktif ~ \nidsubs{i} ~ \kteq ~ {\tt grads\_and\_vars} ~ \ktwhen ~ 1 $\leq$ i $\leq$ k ~ \ktthen\\
  \inden\inden\inden \ktlet ~ \nidsubs{z} ~ \kteq ~ \newid ~ \ktin \\
  \inden\inden\inden ([\nidsubs{z} ~ \oassign ~ \nexprsubs{2i},\\
  \inden\inden\inden \nexprsubs{1} \sparen{\nexprsubs{11} ... \nexprsubs{1n} ~ \op{(\nidsubs{1} \oassign)} \nexprsubs{21} ... \nidsubs{i} \oassign \nidsubs{z} ... \op{(\nidsubs{k} \oassign)} \nexprsubs{2k}},\\
  \inden\inden\inden {\tt global hvd\_broadcast\_done}, \\
  \inden\inden\inden {\tt if not hvd\_broadcast\_done:} [ {\tt hvd.broadcast\_variables([x[1] for x in \nidsubs{z}], root\_rank=0)}, \\
  \inden\inden\inden\inden {\tt hvd.broadcast\_variables(\nidsubs{t}.variables(), root\_rank=0)}, \\
  \inden\inden\inden\inden {\tt hvd\_broadcast\_done = True} ]], \smodenv) \\
  \inden\inden \ktelse \\
  \inden\inden\inden \ktlet ~ \nidsubs{z} ~ \kteq ~ \newid ~ \ktin \\
  \inden\inden\inden ([\nidsubs{z} ~ \oassign ~ \nexprsubs{11},\\
  \inden\inden\inden \nexprsubs{1} \sparen{\nidsubs{z} \nexprsubs{12} ... \nexprsubs{1n} ~ \op{(\nidsubs{1} \oassign)} \nexprsubs{21} ... \op{(\nidsubs{k} \oassign)} \nexprsubs{2k}},\\
  \inden\inden\inden {\tt global hvd\_broadcast\_done}, \\
  \inden\inden\inden {\tt if not hvd\_broadcast\_done:} [ {\tt hvd.broadcast\_variables([x[1] for x in \nidsubs{z}], root\_rank=0)}, \\
  \inden\inden\inden\inden {\tt hvd.broadcast\_variables(\nidsubs{t}.variables(), root\_rank=0)}, \\
  \inden\inden\inden\inden {\tt hvd\_broadcast\_done = True} ]], \smodenv) \\
\end{longtable}
  \caption{GradientTape pattern transform fuinction: the variable broadcast}
  \label{fig:trans:gtape3}
\end{figure}
 
\subsubsection{Transform function for Keras pattern}

In the training codes matching the Keras pattern,
the {\tt Model} instances are created and used to invoke the
training process. The transform function first stores the {\tt Model} instance
variable name in the environment as described in figure \ref{fig:trans:ker}.
The AST pattern guard tests that the given function expression is a subclass
of {\tt tensorflow.keras.Model}, which corresponds to the class constructors
of subclass instances of {\tt tensorflow.keras.Model} class.

\begin{figure}[h]
\begin{longtable}{l}
  \tstmt{\nidsubs{r} \oassign \nexprsubs{1} \sparen{\nexprsubs{11} ... \nexprsubs{1n} ~ \op{(\nidsubs{1} \oassign)} \nexprsubs{21} ... \op{(\nidsubs{k} \oassign)} \nexprsubs{2k}} \optypcomm}{\smodenv} = \\
  \inden \ktelif ~ \nexprsubs{1} ~ \ktsubty ~ {\tt tensorflow.keras.Model} ~ \ktthen\\
  \inden\inden ([\nidsubs{r} \oassign \nexprsubs{1} \sparen{\nexprsubs{11} ... \nexprsubs{1n} ~ \op{(\nidsubs{2} \oassign)} \nexprsubs{21} ... \op{(\nidsubs{k} \oassign)} \nexprsubs{2k}}], \smodenv[\tmodel \nidsubs{r}])\\
\end{longtable}
  \caption{Keras pattern transform function: storing the {\tt Model} instance}
  \label{fig:trans:ker}
\end{figure}

The training process is invoked with {\tt fit} method of the {\tt Model} instance.
For the training codes matching the Keras pattern, the variable broadcast should
be done by adding the {\tt hvd.callbacks.BroadcastGlobalVariablesCallback(root\_rank=0)}
callback to the {\tt fit} method argument.
The transform function utilizes the environment parameter to identify the {\tt fit}
method call and insert the callback object to the argument.
Figure \ref{fig:trans:ker2} describes the transform function that
adds the callback to the {\tt fit} function call expression.

\begin{figure}[h]
\begin{longtable}{l}
  \tstmt{\nexprsubs{1} \sparen{\nexprsubs{11} ... \nexprsubs{1n} ~ \op{(\nidsubs{1} \oassign)} \nexprsubs{21} ... \op{(\nidsubs{k} \oassign)} \nexprsubs{2k}}}{\smodenv} = \\
  \inden \ktelif ~ \nidsubs{m} ~ \kteq ~ \smodenv(\tmodel) ~ \ktand ~ 
          \nexprsubs{1} ~ \kteq ~ {\tt \nidsubs{m}.fit} ~ \ktthen \\
  \inden\inden \ktif ~ \nidsubs{i} ~ \kteq ~ {\tt callbacks} ~ \ktwhen ~ 2 $\leq$ i $\leq$ k ~ \ktthen \\
  \inden\inden\inden ([{\tt callbacks = [hvd.callbacks.BroadcastGlobalVariablesCallback(root\_rank=0)]},\\
  \inden\inden\inden ~ {\tt if hvd.rank() == 0: callbacks.append(\nexprsubs{2i})}, \\
  \inden\inden\inden ~ {\tt \nexprsubs{1} (optim ... \nexprsubs{1n}}
                              \op{(\nidsubs{1} \oassign)} \nexprsubs{21} ... 
                              {\tt callbacks \oassign callbacks} ... 
                              \op{(\nidsubs{k} \oassign)} \nexprsubs{2k}{\tt )}], \smodenv) \\
  \inden\inden \ktelse \\
  \inden\inden\inden ([{\tt callbacks = [hvd.callbacks.BroadcastGlobalVariablesCallback(root\_rank=0)]},\\
  \inden\inden\inden ~ {\tt \nexprsubs{1} (optim ... \nexprsubs{1n}}
                              \op{(\nidsubs{1} \oassign)} \nexprsubs{21} ... 
                              \op{(\nidsubs{k} \oassign)} \nexprsubs{2k}...
                              {\tt callbacks \oassign callbacks} {\tt )}], \smodenv) \\ 
\end{longtable}
  \caption{Keras pattern transform function: adding the broadcast callback}
  \label{fig:trans:ker2}
\end{figure}

The Keras pattern transform function also modifies the {\tt Optimizer} instance.
First, the learning rate parameter of the instance should be scaled by
{\tt hvd.size()}. Second, the instance should be wrapped with
{\tt hvd.Distributed Optimizer} function, similar to the GradientTape pattern's case.
Figure \ref{fig:trans:ker3} describes the transform function modifying
the {\tt Optimizer} instance.

\begin{figure}[h]
\begin{longtable}{l}
  \tstmt{\nidsubs{r} \oassign \nexprsubs{1} \sparen{\nexprsubs{11} ... \nexprsubs{1n} ~ \op{(\nidsubs{1} \oassign)} \nexprsubs{21} ... \op{(\nidsubs{k} \oassign)} \nexprsubs{2k}} \optypcomm}{\smodenv} = \\
  \inden \ktelif ~ \nexprsubs{1} \ktsubty ~ {\tt tensorflow.keras.optimizers.Optimizer} ~ \ktthen\\
  \inden\inden \ktif ~ \nidsubs{i} ~ \kteq ~ {\tt learning\_rate} ~ \ktwhen ~ 1 $\leq$ i $\leq$ k ~ \ktthen\\
  \inden\inden\inden ([\nidsubs{r} \oassign \nexprsubs{1} \sparen{\nexprsubs{11} ... \nexprsubs{1n} ~ \op{(\nidsubs{1} \oassign)} \nexprsubs{21} ... \nidsubs{i} \oassign \nexprsubs{2i} {\tt * hvd.size()}\\
  \inden\inden\inden\inden ... \op{(\nidsubs{k} \oassign)} \nexprsubs{2k}} \optypcomm \\
  \inden\inden\inden {\tt \nidsubs{r} = hvd.DistributedOptimizer(\nidsubs{r})}],
  \smodenv[\optmizer $\mapsto$ \nidsubs{r}])\\
  \inden\inden \ktelse \\
  \inden\inden\inden ([\nidsubs{r} \oassign \nexprsubs{1} \sparen{\nexprsubs{11}
  {\tt * hvd.size()} ... \nexprsubs{1n} ~ \op{(\nidsubs{1} \oassign)}
\nexprsubs{21} ... \op{(\nidsubs{k} \oassign)} \nexprsubs{2k}} \optypcomm, \\
  \inden\inden\inden{\tt \nidsubs{r} = hvd.DistributedOptimizer(\nidsubs{r})}],
  \smodenv[\optmizer $\mapsto$ \nidsubs{r}])\\
\end{longtable}
  \caption{Keras pattern transform function: modifying the {\tt Optimizer} instance}
  \label{fig:trans:ker3}
\end{figure}
