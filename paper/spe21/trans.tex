\section{Code Transformation for Distributed Training}\label{sec:trans}
\subsection{Python Abstract Syntax}\label{sec:pysyn}
Describe the simplified Python abstract syntax here. 
The abstract syntax should be sufficient to describe our code transformation. 
We can attach the full Python abstract syntax as supplementary material.

We first define the abstract syntax for Python programming language.
The syntax of Python is described in the Python Language Reference \cite{pythonref}.
The reference provides a full grammar specification based on the extended PEG,
and detailed explanation of syntatic components in each section.
By examining  
The syntax is composed of three syntactic components: expressions, statements,
and top-level components.

Expressions are parts of the code that evaluates to a value.
Python has 5 kinds of primitive values, 
which are numbers, strings, booleans, the value 'None' and 'Ellipsis.
and has composite types of tuple, list, set, map, and object.
The expression syntax defines ways to build up values
and complex expressions such as operators, comprehension, and function calls. 

Statements are parts of the code that is composed of expressions or other
statements so that they change the program state.
Python statements are categorized into simple statements or compound statements.
Simple statements include assignment statements and import statements.
Note that procedure call is also categorized into simple statements,
as a special case of expression statements.

Compound statements included loops like for and while statements,
and definitions for functions and classes.
Additionally, the with statement is identical to the assignment statement 
in a way that it binds an expression to a name, 
but the statement additionally adds implicit calls
to the object's methods related to initialization and destruction.

Top-level components are representation of the program
in different executionp environment of the interpreter.
For example, module represents a Python code file, composed of
series of statements. 
The full Python abstract syntax is attched in the supplementary material.

\subsection{...}
Describe how to transform TensorFlow ML models to training them on distributed
systems with the horovod framework. We can use formal rules we defined and some
code examples here. We can attach the full transformation rules as
supplementary material as well.  Please write this and some more following
subsections to describe the transformation.
