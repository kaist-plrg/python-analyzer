\section{Introduction}\label{sec:intro}
\begin{itemize}
  \item ML is widely used in various domains these days.
  \item Distributed Training can reduce the training time.
  \item To train ML model on multi-GPUs, code transformation is required but it
    also requires human efforts. The code transformation is simple but only
    some examples describe the transformation.
  \item In this paper, we propose an automated Python code transformation that
    enables TensorFlow ML models to run on multi-GPUs (formal rule of the code
    transformation \& automated code transformation tool).
  \item Describe contributions of this paper.
\end{itemize}

(Main topic sentences for each paragraphs)

Recent advancements in machine learning(ML) opened wide possibility of
applying artificial intelligence in various fields.
(list of area where ML is used. ex. image recognition, autonomous driving)

While the performance of ML models continue to improve,
their growing training time is a major bottleneck in the development process.
This is due to the increase of the size of the training datasets.

Distributed training can reduce the training time 
by parallelizing computation workloads among a number of processors.
(Summarize the distributed training method and its advantage)

Transforming DL model for distributed training involves
manual code rewriting. This requires intensive labor and time costs.

To solve this problem, we utilize code transformation to automate the process.
(define the "code transformation" and how it can "automate" rewriting)

In this paper, we propose an automated Python code transformation that enables
TensorFlow ML models to perform distributed training.

