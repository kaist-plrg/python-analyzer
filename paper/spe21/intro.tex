\section{Introduction}\label{sec:intro}
\begin{itemize}
  \item ML is widely used in various domains these days.
  \item Distributed Training can reduce the training time.
  \item To train ML model on multi-GPUs, code transformation is required but it
    also requires human efforts. The code transformation is simple but only
    some examples describe the transformation.
  \item In this paper, we propose an automated Python code transformation that
    enables TensorFlow ML models to run on multi-GPUs (formal rule of the code
    transformation \& automated code transformation tool).
  \item Describe contributions of this paper.
\end{itemize}

(Main topic sentences for each paragraphs)

Recent advancements in machine learning(ML) have open wide possibility of
applying artificial intelligence in various fields.
(list of area where ML is used. ex. image recognition, autonomous driving)

One major factor in the ML development process is long training time.
Training is a process of updating ML model parameters in a way
that it becomes closer to the optimization point.
This process is inevitably computationally heavy process.
For instance, in deep learning(DL), a ML technique using
neural network-based models, the training process involves
feeding a training data vector into model, forward propagating 

The ML model should be trained against a large amount of training dataset
before it can be accurate and be deployed for real use.

As training time is becoming a major factor in ML development process,
researchers utilize distributed training to reduce the training time.
Distributed training is a technique to parallelize the training computation
workload over multiple processors or accelerators in a distributed system.
As ML technology and its application advance, the number and size of
the training dataset has grown enormously large.

Transforming DL model for distributed training involves
manual code rewriting. This requires intensive labor and time costs.

To solve this problem, we utilize code transformation to automate the process.
(define the "code transformation" and how it can "automate" rewriting)

In this paper, we propose an automated Python code transformation that enables
TensorFlow ML models to perform distributed training.

