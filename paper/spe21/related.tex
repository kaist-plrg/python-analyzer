\section{Related Work}\label{sec:related}

\subsection{Distributed DL frameworks}

%In this work, Horovod library is used to distribute
%DL training codes.
Horovod\cite{sergeev2018horovod} is a popular distributed training library that
supports multiple DL frameworks such as TensorFlow and PyTorch.
Beside Horovod, there are several frameworks and libraries for distributed
training.
TensorFlowOnSpark\cite{tfonspark} is a Python library that
combines TensorFlow with Apache Spark and Hadoop to distribute
DL tasks on server clusters. 
DeepSpeed\cite{deepspeed} is a distributed programming library developed by
Microsoft, built on top of PyTorch\cite{pytorch2019}.  
DeepSpeed supports multiple distributed training methods and features,
including model and pipeline parallelisms. 
TensorFlow officially provides a package of APIs, {\tt tf.distribute}, for
distributed training\cite{tfdistributed}.
The package supports multiple implementations of {\tt tf.distribute.Strategy},
enabling model engineers to perform distributed learning using various
strategies.
To train DL models in distributed environments, model engineers have to choose
a library and manually modify the models following the library's documentation.
Our approach automatically transforms existing DL models into distributed ones,
reducing engineers' burdens of understanding the document and modifying models.

%% explain that none of dist.DL frameworks automatically transform
%None of these frameworks support automatic transformation from single-GPU based
%training to distributed training. 
%The users have to manually modify existing single-GPU-based models in
%accordance with the specific frameworks. 
%In comparison, our work offers an automatic transformation from existing DL
%model to distributed DL model, which reduces programmer burden of manually
%modifying the training code in according to the library specification and use
%case.

%\paragraph{Code transformation.}
\subsection{Code Transformation}

Code transformation is techniques that modify code into a different form.  
Visser devised a taxonomy\cite{Visser2001} that classifies code transformation
techniques into two types: 
\textit{translation}, where input and output code are written in different
languages, and \textit{rephrasing}, where input and output code are written in
the same language.
Our approach belongs to \textit{renovation}, one of the subtypes of
\textit{rephrasing}, which changes the behaviors of input code and generates
output code in the same language.

%% code transformation common usage - compile and optimization
%Code transformation, or program transformation,
%is technique that alters a \textit{source} code
%into a different \textit{target} code. 
%Code transformation is used in variety of area, although
%compilation and optimization is the primary application of
%the technique.
%According to Visser\cite{Visser2001}'s taxonomy, 
%code transformation can be classified into two: \textit{translation},
%where source and target programs are in differene langauge,
%and \textit{rephrasing}, where source and target programs are in
%same language.
%Our approach for automated distributed training is considered to be
%rephrasing transformation, considering it changes a Python code
%into another Python code. 
%In specific, our approach belongs to \textit{renovation} transformation
%according to Visser's taxonomy, considering that
%the transformation alters the behavior of the single-GPU-based
%training code into the distributed training.

% Code Transformation in Python
Researchers have proposed several code transformation techniques in Python
and machine learning domains.
Loulergue and Philippe\cite{Loulergue2020} devised a framework that optimizes
PySke programs by automatically rewriting terms in the programs based on
transformation rules they define.
Haryono et al. \cite{mlcatchup} developed MLCatchUp, a code transformation tool
that enables Python machine learning programs to migrate from deprecated APIs
to new and stable APIs automatically. 
Reed et al. \cite{torchfx} developed {\tt torch.fx}, a Python library for
capturing and transforming PyTorch programs.
Compared to these works, our work targets TensorFlow DL models written in
Python and provides concrete and correct transformation rules for their
distributed training.

%There are several other works on application of program transformation
%in Python and machine learning domain.
%Loulergue and Philippe\cite{Loulergue2020} developed an
%transformation framework that automatically transforms PySke programs, 
%the python skeleton library for parallel programming.
%Haryono et al. \cite{mlcatchup} developed MLCatchUp,
%a tool that automatically updates deprecated APIs in Python machine learning
%programs. Reed et al. \cite{torchfx} developed torch.fx,
%a Python source-to-source transform framework designed to capture
%program structures in Python DL programs and transform them. 
%Compared to these works, our work specifically targets
%TensorFlow DL training codes written in Python,
%and provide concrete and correct transformation rules rather than
%only providing the framework.
