\section{Class hierarchy analysis}\label{sec:cha}
\captionsetup[sub]{font=normal,labelfont={bf,sf}}
\begin{figure}[!ht]
  \centering
  \begin{subfigure}[b]{0.35\textwidth}
    \begin{lstlisting}[style=mpython]
from tensorflow import keras

# `ResNet` inherits `keras.Model`
class ResNet(keras.Model):
    def __init__(self, block_list):
        ...

model = ResNet([2,2,2])
model.fit(x_train, y_train)\end{lstlisting}
    \caption{Single-GPU DL model}
    \label{fig:cha:tfex:a}
  \end{subfigure}
  \hspace{3mm}
  \begin{subfigure}[b]{0.6\textwidth}
    \begin{lstlisting}[style=mpython]
from tensorflow import keras
import horovod.tensorflow.keras as hvd

class ResNet(keras.Model):
    def __init__(self, block_list):
        ...

model = ResNet([2,2,2])

model.fit(
    x_train,
    y_train,
    callbacks=[hvd.callbacks.BroadcastGlobalVariablesCallback(0)])\end{lstlisting}
    \caption{Distributed DL model}
    \label{fig:cha:tfex:b}
  \end{subfigure}

  \caption{Code example of distributing a single-GPU DL model using a user-defined class}
  \label{fig:cha:tfex}
\end{figure}

Figure~\ref{fig:cha:tfex:a} demonstrates how TensorFlow APIs can be used to
create and train a neural network using user-defined classes. 
The key concept in the code is the inheritance of the TensorFlow library class
{\tt keras.Model} by the user-defined class {\tt ResNet}, which allows the user
to access the methods provided by {\tt keras.Model} and use them to train the
neural network.
On line 4 of the code, the user-defined class {\tt ResNet} is defined by
inheriting from {\tt keras.Model}. 
Based-on the Python inheritance mechanism, the {\tt ResNet} class inherits all
the methods and attributes of the {\tt keras.Model} class and can also define
its own methods and attributes. 
The neural network is then created and trained using the {\tt ResNet} class
instead of the {\tt keras.Model} class. 
On line 8, a neural network is created with six blocks containing two layers
each, using the {\tt ResNet} class instead of the {\tt keras.Model} class.
Finally, on line 9, the {\tt fit} method provided by the {\tt keras.Model}
class is called to train the network on the given data.

%Models can utilize TensorFlow APIs via user-defined classes that inherit
%TensorFlow library classes.
%Figure~\ref{fig:cha:tfex:a} shows such a single-GPU model code example.
%On line 4, the user-defined class {\tt ResNet} inherits the TensorFlow library
%class {\tt keras.Model}, the object of which represents a neural network and
%provides methods for training features.
%Using the {\tt ResNet} class instead of the {\tt keras.Model}, the code creates
%and trains a neural network;
%the line 8 creates a neural network with three blocks containing two layers,
%and the line 9 trains the network by calling the API method {\tt fit}
%that the class {\tt keras.Model} provides.

Distributing such models is simple, but we cannot transform them syntactically.
Figure \ref{fig:cha:tfex:b} demonstrates a modification of the model presented
in Figure \ref{fig:cha:tfex:a} to a distributed model. 
The example highlights the importance of recognizing the inheritance
relationship between user-defined classes and the TensorFlow library classes in
identifying the training-related methods. 
The transformation involves adding a keyword argument {\tt callbacks} to the
{\tt fit} method call, as shown on line 13.
However, without recognizing the inheritance relationship between {\tt ResNet}
and {\tt keras.Model}, we cannot identify the training method call on the line
9 of Figure \ref{fig:cha:tfex:a}, and we cannot make the necessary modification
to add the {\tt callbacks} keyword argument.

%The training API pattern identifier and the AST transformer require
%class inheritance relationship to recognize the training-related methods.
%Figure \ref{fig:cha:tfex:b} is an example of modifying the code in
%\ref{fig:cha:tfex:a} into the distributed model.
%The line 9 of \ref{fig:cha:tfex:a} is modified as the lines 10 to 13 of
%\ref{fig:cha:tfex:b}. The modification adds the keyword argument
%{\tt callbacks} to the {\tt fit} method call.
%To apply this modification, we must first recognize that the {\tt fit} method
%in the line 9 of \ref{fig:cha:tfex:a} is inherited from the
%{\tt keras.Model} class.
%Without recognizing the inheritance relation between {\tt ResNet} and
%{\tt keras.Model}, we cannot identify the training method call
%in the line 9.

The class hierarchy analysis is an essential pre-analysis step that we employ
to solve the problem of identifying the training-related methods in DL models.
The class hierarchy analysis is a static analysis technique that identifies the
inheritance relationship between the classes in the code. 
By applying the class hierarchy analysis on the input DL models, we can
identify which user-defined classes inherit TensorFlow library classes, and
therefore correctly identify the training-related method calls inherited from
TensorFlow library classes.
In the code example in Figure \ref{fig:cha:tfex:a}, for instance, the class
hierarchy analyzer reads the class definition on line 4 to conclude that the
class {\tt ResNet} inherits the class {\tt keras.Model}. 
The \tapi~takes this information so that it recognizes the {\tt fit} method
call as the call to the training method provided by the {\tt keras.Model} and
selects appropriate transformation rules for the training pattern.
The information is also sent to \atran~to identify the training method call
statement and apply the transformation rules on it.

%We employ the class hierarchy analysis as a pre-analysis to solve this problem.
%\textit{Class hierarchy analysis} is a static analysis technique that identifies
%inheritance relation between the classes in the code.
%By applying class hierarchy analysis on the input DL models,
%we can identify which user-defined classes inherit TensorFlow library classes,
%and therefore correctly identify the training-relate
%method calls inherited from TensorFlow library classes.
%In the figure \ref{fig:cha:tfex:a}, for instance,
%the class hierarchy analyzer reads the line 4 to conclude that the class
%{\tt ResNet} inherits the class {\tt keras.Model}.
%This information is sent to the training API pattern identifier and
%AST transformer so that they can recognize the {\tt fit} method in line 9
%as the call to the training method.

%\begin{figure}[!ht]
%  \centering
%  \begin{subfigure}[t]{0.35\textwidth}
%    \begin{lstlisting}[style=mpython]
%from tensorflow import keras
%
%# `ResNet` inherits `keras.Model`
%class ResNet(keras.Model): 
%    def __init__(self, block_list):
%        ...
%
%model = ResNet([2,2,2])
%model.fit(x_train, y_train)\end{lstlisting}
%    \caption{Single-GPU model using an user-defined class}
%    \label{fig:cha:tfex:a}
%  \end{subfigure}
%  \hspace{3mm}
%  \begin{subfigure}[t]{0.6\textwidth}
%    \begin{lstlisting}[style=mpython]
%from tensorflow import keras
%import horovod.tensorflow.keras as hvd
%
%class ResNet(keras.Model):
%    def __init__(self, block_list):
%        ...
%
%model = ResNet([2,2,2])
%
%model.fit(
%    x_train, 
%    y_train,
%    callbacks=[hvd.callbacks.BroadcastGlobalVariablesCallback(0)])\end{lstlisting}
%    \caption{Distributed model using an user-defined class}
%    \label{fig:cha:tfex:b}
%  \end{subfigure}
%
%  \caption{Code example of distributing a model using an user-defined class}
%  \label{fig:cha:tfex}
%\end{figure}
%
%
%%The first step for the automated code transformation identifies TensorFlow APIs
%%used in models to apply a different transformation rule depending on the API
%%pattern. 
%%However, we cannot perform the identification syntactically, because models can
%%call the TensorFlow APIs via user-defined classes that inherit TensorFlow
%%library classes.
%%Figure~\ref{fig:cha:tfex:a} shows such a single-GPU model code example.
%%On line 4, the user-defined class {\tt ResNet} inherits the TensorFlow library
%%class {\tt keras.Model}, the object of which represents a neural network and
%%provides methods for training features.
%%Using the {\tt ResNet} class instead of the {\tt keras.Model}, the code creates
%%and trains a neural network;
%%the line 8 creates a neural network with three blocks containing two layers,
%%and the line 9 trains the network by calling the API method {\tt fit}
%%that the class {\tt keras.Model} provides.
%%Figure~\ref{fig:cha:tfex:b} shows the distributed model of the single-GPU
%%model.
%%The transformation first identifies a statement calling the {\tt fit} API
%%method of the {\tt keras.Model} class and then passes an additional {\tt
%%callbacks} argument to the method as shown on line 13.
%%The transformation is simple in the sense that it requires only the
%%modification of the {\tt fit} method call statement.
%%However, we cannot syntactically check whether the {\tt fit} method is the API
%%method of the {\tt keras.Model} class or a user-defined method having the same
%%name, because the {\tt model} object is an instance of the user-defined class
%%{\tt ResNet}, not the TensorFlow library class {\tt kerase.Model}.
%
%
%To solve the problem, we design and implement the class hierarchy analysis for
%Python as a pre-analysis of the transformation.
%The class hierarchy analysis is a static analysis technique that identifies
%inheritance relation between the classes in the code.
%By applying the class hierarchy analysis on the input DL models,
%we can identify which user-defined classes inherit TensorFlow library classes,
%and therefore correctly identify the training-relate
%method calls inherited from TensorFlow library classes.
%
%
%%By inheriting the {\tt keras.Model} class, the {\tt ResNet} class
%%defines a new model structure while reusing the methods 
%%defined in the {\tt keras.Model} class.
%%Then line 9 calls the {\tt fit} method, inherited from the {\tt keras.Model} 
%%class. 
%%The {\tt fit} methods trains the model with the training dataset given as
%%arguments.
%
%%We employ the class hierarchy analysis as a pre-analysis to solve this problem. 
%%\textit{Class hierarchy analysis} is a static analysis technique that identifies
%%inheritance relation between the classes in the code.
%By applying class hierarchy analysis on the input DL models,
%we can identify which user-defined classes inherit TensorFlow library classes,
%and therefore correctly identify the training-relate
%method calls inherited from TensorFlow library classes.
%In the figure \ref{fig:cha:tfex:a}, for instance, 
%the class hierarchy analyzer reads the line 4 to conclude that the class
%{\tt ResNet} inherits the class {\tt keras.Model}.
%This information is sent to the training API pattern identifier and 
%AST transformer so that they can recognize the {\tt fit} method in line 9 
%as the call to the training method. 
%
%%To correctly identify TensorFlow APIs used in models, the {\sc Training API
%%Pattern Identifier} requires inheritance relations between classes.
%%Figure \ref{fig:cha:tfex:b} is an example of modifying the code in 
%%\ref{fig:cha:tfex:a} into the distributed model.
%%The line 9 of \ref{fig:cha:tfex:a} is modified as the lines 10 to 13 of 
%%\ref{fig:cha:tfex:b}. The modification adds the keyword argument 
%%{\tt callbacks} to the {\tt fit} method call. 
%%To apply this modification, we must first recognize that the {\tt fit} method
%%in the line 9 of \ref{fig:cha:tfex:a} is inherited from the
%%{\tt keras.Model} class.
%%Without recognizing the inheritance relation between {\tt ResNet} and
%%{\tt keras.Model}, we cannot identify the training method call 
%%in the line 9.  
%
%%the {\sc Training API Pattern Identifier} and the {\sc AST Transformer} require
%%class inheritance relationship to recognize the training-related methods.
%%Figure \ref{fig:cha:tfex:b} is an example of modifying the code in 
%%\ref{fig:cha:tfex:a} into the distributed model.
%%The line 9 of \ref{fig:cha:tfex:a} is modified as the lines 10 to 13 of 
%%\ref{fig:cha:tfex:b}. The modification adds the keyword argument 
%%{\tt callbacks} to the {\tt fit} method call. 
%%To apply this modification, we must first recognize that the {\tt fit} method
%%in the line 9 of \ref{fig:cha:tfex:a} is inherited from the
%%{\tt keras.Model} class.
%%Without recognizing the inheritance relation between {\tt ResNet} and
%%{\tt keras.Model}, we cannot identify the training method call 
%%in the line 9.  
%
%%We employ the class hierarchy analysis as a pre-analysis to solve this problem. 
%%\textit{Class hierarchy analysis} is a static analysis technique that identifies
%%inheritance relation between the classes in the code.
%%By applying class hierarchy analysis on the input DL models,
%%we can identify which user-defined classes inherit TensorFlow library classes,
%%and therefore correctly identify the training-relate
%%method calls inherited from TensorFlow library classes.
%%In the figure \ref{fig:cha:tfex:a}, for instance, 
%%the class hierarchy analyzer reads the line 4 to conclude that the class
%%{\tt ResNet} inherits the class {\tt keras.Model}.
%%This information is sent to the training API pattern identifier and 
%%AST transformer so that they can recognize the {\tt fit} method in line 9 
%%as the call to the training method. 
