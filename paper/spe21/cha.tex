\section{Class hierarchy analysis}\label{sec:cha}

\subsection{Motivation}

To correctly transform TensorFlow DL model codes,
one must identify the TensorFlow training APIs used in the training code.
This includes creating and using instances of TensorFlow classes or
calling methods of TensorFlow methods. 
For instance, a developer can define a DL model by creating an instance of
{\tt tf.keras.Sequential} class and train it by calling the
{\tt fit} method on the model instance.
Figure \ref{fig:cha:tfex} shows an example of using
TensorFlow classes and methods.

\begin{figure}[h]
\begin{lstlisting}[language=Python]
import tensorflow as tf
import tensorflow.keras as keras

model = keras.Sequential(...) # define a model
model.fit() # train the model by calling 
\end{lstlisting}
  \caption{Training code example of using TensorFlow library class and method}
  \label{fig:cha:tfex}
\end{figure}

In practice, developers may want to extend or modify original behaviors
of TensorFlow library classes or methods.  
Developers can define a new class that inherits TensorFlow library classes
and override methods to extend their behaviors. 
These \textit{user-defined training classes}
should also be identified to prepare a correct transformation rule 
for different implementations.  However, using only syntactic information 
cannot fully identify user-defined training classes
as different developers use different identifiers for naming classes.
Figure \ref{fig:cha:ex} shows an instance of using the
user-defined training class, the {\tt ResNet} class inheriting
the {\tt keras.models.Model} class.
The selection of the name {\tt ResNet} here is completely arbitrary;
it can be any other name that cannot be fully recognized by
pre-defined syntactic rules.
Identifying the method name {\tt fit} is also not an correct method.
Line 12 of figure \ref{fig:cha:ex} defines another class {\tt Something}
that has a method {\tt fit} but is not related to the training.
Using the method name to identify the traninig API will lead to
incorrectly assuming that the instance {\tt something} in line 20
is the TensorFlow model instance, hence resulting in incorrect transformation.

\begin{figure}[h]
  \begin{lstlisting}[language=Python]
import tensorflow as tf
import tensorflow.keras as keras

# user-defined training class
class ResNet(keras.models.Model): 
    def __init__(self, layer_num, **kwargs):
        pass
    def call(self, inputs):
        pass

# arbitrary non-training class 
class Something():
    def fit(optimizer, y):
        print("hello, world!") 

if __name__ = '___main__': 
    # actual DL model instance
    model = ResNet(50)
    # object not related to training
    something = Something()

    model.build(input_shape=(None,) + c.input_shape) 
    optim = tf.keras.optimizers.SGD()

    # training-related method call, which should be transformed
    model.fit(optimizer=optim, train_data) 
    # method call not related to the training  
    something.fit(optimizer=optim, train_data)
\end{lstlisting}
\caption{Training code example of using a user-defined class}
\label{fig:cha:ex}
\end{figure}

To solve this problem, we devised to pre-analyze the class inheritance
relation between TensorFlow library classes and user-defined classes. 
We utilize a \textit{class hierarchy analysis}, a type of static analysis
that identifies   


\subsection{Class Hierarchy Analyzer}



Figure \ref{fig:cha:ex} illustrates an example of a training code
using the user-defined class. 
Line 4 defines a new class {\tt ResNet} that inherits the  
{\tt keras.models.Model} class. The instance of {\tt ResNet}, {\tt model}
is created at line 11. Although the {\tt ResNet} class does not define
the method {\tt fit}, line 14 calls the {\tt fit} method to train the model.
Because the {\tt fit} method is inherited from the {\tt Model} class,
the {\tt ResNet} instance can invoke the training process with
{\tt model.fit}. The inherited model {\tt fit}
automatically repeats the gradient descent with
given model computation and training data. 

The class hierarchy analyzer(CHA) module analyzes the
subclass relationship between the 
user-defined classes and TensorFlow classes.
The module returns a class hierarchy graph(CHG) as an output.
A CHG is a directed graph structure that represents the class hierarchy.
Each node in the graph represents a class,
and each directed edge represents a class inheritance,
where the start node indicates the child class
and the end node indicates the parent class.
Consider the example in figure \ref{fig:cha:pythonex}.
Class B is a subclass of class A, and class C is a subclass of 
class B.
This subclass relationship is represented as a CHG with three nodes,
each corresponding to a class A, B, or C.
Suppose node A corresponds to class A, node B corresponds to
class B, and node C corresponds to class C.
Then the CHG includes two directed edges, from A to B and from B to C,
each representing the inheritance relationship of the classes.
Identifying a subclass relationship between the class A and class C
reduces to finding a (direct) path between node A to node C in the CHG;
in the figure \ref{fig:cha:pythonex} case, the CHG contains a
path from node A to node C, therefore class C is a subclass
of class A.


The CHA module gets a model code package as an input
and creates the model package CHG as an output.
The analysis iterates over the files in the package
and extracts every subclass information available in each file.
New Python classes are defined with the {\tt class} statements.
In {\tt class} statements, the newly defined class can
inherit another class by specifying the class expression
in the argument position.
The analyzer finds all {\tt class} statements from the module AST,
adds the CHG node for the new class. The analyzer also adds an edge
to the parent class node if the {\tt class} statement specifies an inheritance.


