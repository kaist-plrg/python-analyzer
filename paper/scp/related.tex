\section{Related Work}\label{sec:related}

\subsection{Distributed DL frameworks}

%In this work, Horovod library is used to distribute
%DL training codes.
Horovod~\cite{sergeev2018horovod} is a popular distributed training library that
supports multiple DL frameworks such as TensorFlow and PyTorch.
Besides Horovod, there are several frameworks and libraries for distributed
training.
TensorFlowOnSpark~\cite{tfonspark} is a Python library that
combines TensorFlow with Apache Spark and Hadoop to distribute
DL tasks on server clusters. 
DeepSpeed~\cite{deepspeed} is a distributed programming library developed by
Microsoft, built on top of PyTorch~\cite{pytorch2019}.  
DeepSpeed supports multiple distributed training methods and features,
including model and pipeline parallelisms. 
TensorFlow officially provides a package of APIs, {\tt tf.distribute}, for
distributed training~\cite{tfdistributed}.
The package supports multiple implementations of {\tt tf.distribute.Strategy},
enabling model engineers to perform distributed learning using various
strategies.
To train DL models in distributed environments, model engineers have to choose
a library and manually rewrite the models following the library's documentation.
Our approach automatically transforms existing DL models into distributed ones,
reducing engineers' burdens of understanding the document and modifying models.

%% explain that none of dist.DL frameworks automatically transform
%None of these frameworks support automatic transformation from single-GPU based
%training to distributed training. 
%The users have to manually modify existing single-GPU-based models in
%accordance with the specific frameworks. 
%In comparison, our work offers an automatic transformation from existing DL
%model to distributed DL model, which reduces programmer burden of manually
%modifying the training code in according to the library specification and use
%case.

\begin{inred}

\subsection{Automatically Distributing Deep Learning Models}

Several works proposed techniques on automatically optimizing the
distributed training of DL models.
A line of work searches for optimal parallelization plans
before runtime of the DL model training codes.
Megatron-LM~\cite{megatron-lm} is one of the earliest works on automatically 
finding efficient parallelization plans of DNN model training. 
The Megatron-LM framework distributes operations inside the layers of 
transformer networks by adding synchronization primitives.
Pipedream~\cite{pipedream} is an asynchronous distributed training framework 
that improves parallel training throughput by adding pipelining and intra-batch parallelism. 
PipeDream also minimizes communication cost by partitioning the training computation based on a profiling run on a single GPU.
Alpa~\cite{alpa} is an automatic model-parallel training framework that 
organizes different parallelization techniques into a hierarchy 
and maps them to the hierarchical structure of compute devices. 
This allows Alpa to optimize plans for both inter-operator parallelization and intra-operator parallelization.
Lin et al.~\cite{nnscaler} introduce new primitive operators 
that allow domain experts to compose their own search space, 
then the proposed framework generates an efficient parallelizationplan for the DNN model training.
AutoDDL~\cite{autoddl} adds new tensor operators to expand the search space for parallelization strategies. 
The framework uses an analytical performance model with a Coordinate Descent-based search algorithm to optimize the communication cost.

Another line of work focuses on optimizing the resource usage
during the training runtime. 
DAPPLE~\cite{dapple} is a synchronous distributed training framework 
that combines data and pipeline parallelism for large DNN models, 
which finds the optimal parallelization plan with a novel parallelization strategy planner 
and a new runtime scheduling algorithm to reduce memory usage.
Tiresias~\cite{tiresias} is a GPU cluster manager 
that efficiently schedules DL training jobs to reduce their job completion times. 
Because it is difficult to predict when the DL jobs completes, 
the authors propose two scheduling algorithms tailored to minimize the average job completion times.

Compared to these works, 
our work's main technique is to optimize the DL model training by
automatically transforming single-GPU-based training codes into distributed training codes.
Aforementioned works require developers to manually write distributed training codes
according to the proposed framework.
Our work utilizes the Horovod library’s data-parallel distributed training to 
distribute existing single-GPU-based DL training codes without rewriting them.
While previous works’ main technique is to distribute low-level, 
primitive compute operation over multiple GPUs, 
our main technique is to transform a single-GPU-based DL training code 
to a distributed version in Python source code level. 
By reusing the Horovod library and converting the training code on the source code level, 
we allow non-expert developers to easily distribute the training code without manually rewriting the original training code.



\end{inred}

%\paragraph{Code transformation.}
\subsection{Code Transformation}

Code transformation is techniques that modify code into a different form.  
Visser devised a taxonomy~\cite{visser2001survey} that classifies code transformation
techniques into two types: 
\textit{translation}, where input and output code are written in different
languages, and \textit{rephrasing}, where input and output code are written in
the same language.
Our approach belongs to \textit{renovation}, one of the subtypes of
\textit{rephrasing}, which changes the behaviors of input code and generates
output code in the same language.

%% code transformation common usage - compile and optimization
%Code transformation, or program transformation,
%is technique that alters a \textit{source} code
%into a different \textit{target} code. 
%Code transformation is used in variety of area, although
%compilation and optimization is the primary application of
%the technique.
%According to Visser\cite{Visser2001}'s taxonomy, 
%code transformation can be classified into two: \textit{translation},
%where source and target programs are in differene langauge,
%and \textit{rephrasing}, where source and target programs are in
%same language.
%Our approach for automated distributed training is considered to be
%rephrasing transformation, considering it changes a Python code
%into another Python code. 
%In specific, our approach belongs to \textit{renovation} transformation
%according to Visser's taxonomy, considering that
%the transformation alters the behavior of the single-GPU-based
%training code into the distributed training.

% Code Transformation in Python
Researchers have proposed several code transformation techniques in Python.
Loulergue and Philippe~\cite{loulergue2019automatic} devised a framework that optimizes
PySke programs by automatically rewriting terms in the programs based on
transformation rules they define.
Haryono et al.~\cite{haryono2021mlcatchup} developed MLCatchUp, a code transformation tool
that enables Python machine learning programs to migrate from deprecated APIs
to new and stable APIs automatically. 
\begin{inred}
In Zhang et al. \cite{python-idiomatic}, the authors develop AST-rewriting 
operations to detect non-idiomatic code for each pythonic idiom and refactor 
non-idiomatic Python codes into idiomatic codes.
Rózsa et al. \cite{dont-diy}  introduces a framework 
which transforms traditional conditional branching to structural pattern matching 
without changing the original behavior. 
This improves the readability and maintainability of codes written in legacy Python.
Tangent \cite{tangent} is a new library that transforms a subset of Python and NumPy code 
to perform automatic differentiation on a source code level.
Compared to these works, our work targets TensorFlow DL models written in
Python and provides concrete and correct transformation rules for their
distributed training.
To our knowledge, our work is the first work that utilizes Python code transformation technique to 
automatically distribute single-GPU-based DL training codes.
\end{inred}{

%There are several other works on application of program transformation
%in Python and machine learning domain.
%Loulergue and Philippe\cite{Loulergue2020} developed an
%transformation framework that automatically transforms PySke programs, 
%the python skeleton library for parallel programming.
%Haryono et al. \cite{mlcatchup} developed MLCatchUp,
%a tool that automatically updates deprecated APIs in Python machine learning
%programs. Reed et al. \cite{torchfx} developed torch.fx,
%a Python source-to-source transform framework designed to capture
%program structures in Python DL programs and transform them. 
%Compared to these works, our work specifically targets
%TensorFlow DL training codes written in Python,
%and provide concrete and correct transformation rules rather than
%only providing the framework.
