\documentclass[preprint, 12pt]{elsarticle}% Math and Physical Sciences Reference Style

% \articletype{Research article}%

% \received{26 April 2016}
% \revised{6 June 2016}
% \accepted{6 June 2016}

\raggedbottom

\usepackage{graphicx}%
\usepackage{multirow}%
\usepackage{amsmath,amssymb,amsfonts}%
\usepackage{amsthm}%
\usepackage{mathrsfs}%
\usepackage[title]{appendix}%
\usepackage{xcolor}%
\usepackage{textcomp}%
\usepackage{manyfoot}%
\usepackage{booktabs}%
\usepackage{algorithm}%
\usepackage{algorithmicx}%
\usepackage{algpseudocode}%
\usepackage{listings}%
\usepackage{tabularx}

\input{macro}
\usepackage{url}
\usepackage[T1]{fontenc}
\lstdefinestyle{mpython}{
  language=Python,
  upquote=true,
  morekeywords={with,as},
  emphstyle=\color{blue},
  emph={True,False},
  deletekeywords=[2]{compile},
}

\journal{Science of Computer Programming}

\begin{document}

\begin{frontmatter}

\title{Automated Code Transformation for Distributed Training of TensorFlow Deep Learning Models}

\author[inst1]{Yusung Sim}

\author[inst1]{Wonho Shin}

\author[inst2]{Sungho Lee\corref{cor1}}

\cortext[cor1]{Corresponding author}

%\authormark{AUTHOR ONE \textsc{et al}}

%\affil[1]{\orgdiv{School of Computing}, \orgname{KAIST}, \orgaddress{\state{Daejeon}, \country{Republic of Korea}}}

\affiliation[inst1]{organization={KAIST},%Department and Organization
            addressline={291 Daehak-ro, Yuseong-gu}, 
            city={Daejeon},
            postcode={34141}, 
            country={Republic of Korea}}

%\address[2]{\orgdiv{School of Computing}, \orgname{KAIST}, \orgaddress{\state{Daejeon}, \country{South Korea}}}

% \affil[2]{\orgdiv{Department of Computer Science and Engineering}, \orgname{Chungnam National University}, \orgaddress{\state{Daejeon}, \country{Republic of Korea}}}
\affiliation[inst2]{organization={Chungnam National University},%Department and Organization
            addressline={99 Daehak-ro, Yuseong-gu}, 
            city={Daejeon},
            postcode={34141}, 
            country={Republic of Korea}}

% \corres{Sungho Lee, Department of Computer Science and Engineering, Chungnam National University, 99 Daehak-ro, Yuseong-gu, Daejeon 34134, Republic of Korea. \email{eshaj@cnu.ac.kr}}

%\fundingAgency{National Research Foundation of Korea}
%\fundingNumber{2021R1F1A1051310}
%\presentaddress{This is sample for present address text this is sample for present address text}


\begin{abstract}
Distributed training of deep learning models reduces training time by
parallelizing training workloads across multiple GPUs.
Distributed training frameworks, such as Horovod and DeepSpeed, provide APIs,
and model engineers rewrite deep learning models using the APIs to parallelize
their training.
However, the rewriting is time-consuming and labor-intensive because it
requires engineers to read and understand documents and examples of the
frameworks as well as manual efforts to rewrite code.

In this paper, we propose an automated code transformation approach that
transforms TensorFlow deep learning models designed for non-distributed
training to models training on multiple GPUs with the Horovod framework.
We closely inspect the Horovod document and code examples and identify four
common training patterns of TensorFlow deep learning models.
Then, we formalize code transformation rules for each training pattern. 
Using the rules, we implement an automated code transformation tool that takes
a TensorFlow deep learning model written in Python and rewrites it with the
Horovod APIs for distributed training. 
Our evaluation shows that the tool correctly transforms 15 out of 16
open-source TensorFlow deep learning models.
We believe that our approach significantly reduces manual efforts to
parallelize training of existing TensorFlow deep learning models.
\end{abstract}

%%Research highlights
%\begin{highlights}
%\item Research highlight 1
%\item Research highlight 2
%\end{highlights}

\begin{keyword}
%% keywords here, in the form: keyword \sep keyword
machine learning \sep distributed training \sep code transformation \sep Python
%% PACS codes here, in the form: \PACS code \sep code
%\PACS 0000 \sep 1111
%% MSC codes here, in the form: \MSC code \sep code
%% or \MSC[2008] code \sep code (2000 is the default)
%\MSC 0000 \sep 1111
\end{keyword}

% \keywords{machine learning, distributed training, code transformation, Python}

% \maketitle

\end{frontmatter}
%\footnotetext{\textbf{Abbreviations:} ANA, anti-nuclear antibodies; APC, antigen-presenting cells; IRF, interferon regulatory factor}

\section{Introduction}\label{sec:intro}

Recent advancements in machine learning(ML) have open the wide possibility of
applying artificial intelligence in various fields.
(list of area where ML is used. ex. image recognition, autonomous driving)

One major factor to consider in ML development is training time.
While training is an essential part of the ML development,
the process requires heavy computational workloads.
According to the analysis by S. Bianco et al. \cite{bianco2018benchmark},
majority of the image recognition models including ResNet, VGG, SENet, etc.
require over 5 GFLOPs for a single forward propagation.
The training process is repeatition of training steps until convergence,
which each of them is composed of a forward and a backward propagation.
Considering that thousands of training data is feed into the model
for the convergence, the training process takes the longest time
in the whole development process. This leads to the conclusion that
reducing training time is the most efficient optimization according to
the Amdahl's law.

In one of the efforts to reduce training time, 
researchers utilize \textit{distributed training}.
Distributed training is a technique to parallelize the training computation
workload over multiple GPUs.
By taking advantage of parallelism, distributed training enables researchers
to spend less time in training while preserving accuracy.
Using multiple GPUs or TPUs to train the model is already adopted
in various works \cite{brown2020gpt-3} \cite{silver2017alphazero}
\cite{zhang2019distrspeech} \cite{tian2020distrwebattack}.
As ML models are becoming more complex and training dataset are growing,
the need for distributing the training process in ML research is inevitable.

Transforming an ML model for distributed training involves code rewriting.
Until now, developers have manully rewrite the model codes.
This is time-consuming and labor-intensive tasks for developers.
In addiiton, developers need to understand and locate 
specific components involved in the model training.
This requires the developer to fully understand the library APIs,
which is a difficult challenge.

In this paper, we propose an 
\textit{automated code transformation for distributed training}.
The transformation
converts the single-GPU based TensorFlow models
into the multi-GPU based models, so that the model can be trained
over a distributed systm.
We first formally define the code transformation from
single-GPU based ML model code to distributed training model code.
Then the transformation is implemented in actual software together with
adequate pre-analysis that provides information used in the transformation.
The transformation software is evaluated by comparing against 
examples of distributed training model codes, 
each including an original single-GPU based model code 
and a manually transformed distributed model code.

The contributions of this paper are as follows:

\begin{itemize}
  \item We formalize the code transformation for distributed ML training
        of TensorFlow models. We first formally define the code transformation
        as functions from AST to AST. Then we provide transform function
        definitions for ouistributed ML model code.
  \item We present the distributed code transformation tool, which implements
        the code transformation functions. We evaluate the tool's performance
        by applying transformation to TensorFlow example model codes.
\end{itemize}

\section{Background}\label{sec:background}

\subsection{TensorFlow Deep Learning Models}
\begin{figure}[ht!]\centering
\includegraphics[width=0.7\textwidth]{mnist_model.pdf}
  \caption{The neural network with a hidden layer and an output layer}
\label{fig:back:model}
\end{figure}

This section describes two different forms of TensorFlow DL models written in
Python.
TensorFlow provides two major version libraries, TensorFlow 1.x published in
2016 and TensorFlow 2.x published in 2019. 
DL models are significantly differ in their forms depending on which library
they use. 
On TensorFlow 1.x, neural networks are manually constructed as computational
graphs using tensor variables and operations, and the networks are lazily
executed for both their training and inference on an encapsulated environment,
called {\it session}.
%The first version is TensorFlow 1.x published in 2016.
%TensorFlow 1.x provides APIs for defining tensor variables and operations
%between them.
%Developers manually define the model structure, the model operations 
%and training process using the TensorFlow 1.x APIs.
%On TensorFlow 1.x, developers need to manually define model structures, model
%operations, and their training process using APIs for defining tensor variables
%and operations between them.
%The second version is TensorFlow 2.x published in 2019.
On the other hand, TensorFlow 2.x supports the eager execution by which all
tensor operations are executed as they occur in code. 
With the eager execution feature, developers do not need to construct
computational graphs and use encapsulated environments anymore.
%developers to use plain Python syntax to define the model
%operations and the training process.
In addition, TensorFlow 2.x integrates with the Keras library, a layer-based
deep learning model library that provides a convenient interface to construct
neural networks.
%As a result, the models written in TensorFlow 1.x and TensorFlow 2.x are
%significantly differ in their forms. 

Figure~\ref{fig:back:model} illustrates an example neural network that
classifies input images into ten categories.
In the figure, vertical bars denote vectors of layers, circles denote data of
vectors, and direct edges denote data dependencies from source to destination
in vector operations.
%The model will get an input image of 784 pixels and classify it into one of ten
%output classes.
The network consists of three layers: an input layer, an output layer, and a
hidden layer between the two layers.
Each layer stores data into a vector, and the vector of a layer mutates into a
vector of another layer via vector operations.  
In the network, the input layer has a vector of length 784, and the data in the
vector is the pixels of an input image.
The hidden layer is parametrized by the two-dimensional weight matrix $W_1$ of
size 784 $\times$ 100 and the bias vector $b_1$ of length 100.
The vector of the hidden layer is computed by first multiplying the input layer
vector $x$ with the weight matrix $W_1$, adding the bias vector $b_1$, and
finally applying the ReLU activation function that makes the result as non-zero
values.
The output layer is parametrized by the two-dimensional weight matrix $W_2$ of
size 100 $\times$ 10 and the bias vector $b_2$ of length 10.
The vector of the output layer is computed by first multiplying the hidden
layer vector $h$ with the weight matrix $W_2$, adding the bias vector $b_2$,
and finally applying the Softmax activation function that converts the ten data
to a probability distribution of the ten categories.
The weight matrices and the bias vectors in the network are model parameters,
and the training phase adjusts the model parameters again and again to classify
input images correctly.
%The model structure is defined in terms of layers and operation between them.
%The layers are tensors of real values, which represent the signals and data.
%The model in the figure \ref{fig:back:model} has three layers;
%the input layer, the hidden layer, and the output layer.
%The layers are densely connected, which means that
%the layer outputs are computed by the linear transformation with 
%the weight matrices and the bias vectors, 
%and the non-linear activation functions. 
%The hidden layer is parametrized by the weight matrix
%$W_1$ of size (784, 100) and the bias vector $b_1$ of length 100.
%The hidden layer is computed by first multiplying the input layer vector $x$
%with the weight matrix $W_1$, adding the bias vector $b_1$, and finally applying
%the ReLU activation function to the result.
%The output layer is parametrized by the weight matrix $W_2$ of size (100, 10)
%and the bias vector $b_2$ of length 10.
%The output layer is computed by first multiplying the hidden layer vector $h$
%with the weight matrix $W_2$, adding the bias vector $b_2$, and finally applying
%the Softmax activation function to the result.
%The weight matrices and the bias vectors are called the model parameters 
%as their values are updated during the training process.
%The hidden layer is a vector of length 100.
%The output layer is a vector of length 10, and each vector element represents
%the probability of the input image classified into the corresponding class.

%We describe TensorFlow 1.x and 2.x versions with code examples that define
%the same model and the same training process.
%The code examples define the neural network model illustrated
%in the figure \ref{fig:back:model}.
%The model will get an input image of 784 pixels and classify it into one of ten
%output classes.
 
%The model structure is defined in terms of layers and operation between them.
%The layers are tensors of real values, which represent the signals and data.
%The model in the figure \ref{fig:back:model} has three layers;
%the input layer, the hidden layer, and the output layer.
%The input layer is a vector of length 784, which represents the input image
%of 784 pixels.
%The hidden layer is a vector of length 100.
%The output layer is a vector of length 10, and each vector element represents 
%the probability of the input image classified into the corresponding class.
%The layers are densely connected, which means that
%the layer outputs are computed by the linear transformation with 
%the weight matrices and the bias vectors, 
%and the non-linear activation functions. 
%The hidden layer is parametrized by the weight matrix
%$W_1$ of size (784, 100) and the bias vector $b_1$ of length 100.
%The hidden layer is computed by first multiplying the input layer vector $x$
%with the weight matrix $W_1$, adding the bias vector $b_1$, and finally applying
%the ReLU activation function to the result.
%The output layer is parametrized by the weight matrix $W_2$ of size (100, 10)
%and the bias vector $b_2$ of length 10.
%The output layer is computed by first multiplying the hidden layer vector $h$
%with the weight matrix $W_2$, adding the bias vector $b_2$, and finally applying
%the Softmax activation function to the result.
%The weight matrices and the bias vectors are called the model parameters 
%as their values are updated during the training process.

%The model is trained by the gradient descent algorithm. 
%The gradient descent algorithm is an iterative optimization algorithm
%that computes the gradients of model parameters to update their values toward
%local minimum of the loss function.
%The loss function defines the metric of the difference between the model 
%output and answer label.
%Thus, approaching the local minumum of the loss function means that the
%the gradient descent trains the model to return the output similar to the
%correct answer.
%In DL model training, the gradient descent is repeated against several
%training batches. Given a training input batch, the algorithm first computes
%the model loss with forward propagation then computes the gradients of the
%model parameters with backpropagation. The algorithm applies the gradients
%to update the model parameters, thus approaching to the minumum loss.

\begin{figure}[ht!]
\lstinputlisting[style=mpython]
{tensorflow1_mnist.py}
  \caption{TensorFlow 1.x model example}
\label{fig:back:tf1}
\end{figure}

Figure~\ref{fig:back:tf1} shows a TensorFlow 1.x model for the example neural
network with the gradient descent training algorithm.
First, the lines 5 to 14 define the network structure and the operations
between the layers.
The lines 5 and 6 first create two placeholder variables, {\tt x} and {\tt y},
where {\tt x} is a vector storing the pixel data of an input image, and {\tt y}
is a vector storing the answer for the classification of the input image. 
The lines 8 to 10 define the hidden layer.
The line 8 creates a randomly initialized weight matrix {\tt W\_1}, and the
line 9 creates a zero-initialized bias vector {\tt b\_1}.
Then, the matrix and the bias vector are wrapped with the {\tt Variable} API.
The {\tt Variable} API creates a model parameter of which internal values are
modifiable during the training process.
The line 10 defines the operation of the hidden layer. It first multiplies the
input vector {\tt x} with the weight matrix {\tt W\_1}, then adds the bias {\tt
b\_1}, and finally applies the {\tt relu} activation fuction. 
Note that the line does not actually perform the operation but only defines how
the input data mutates into the hidden layer's data during executing the
network.
The lines 12 to 14 define the output layer.
Similar to the lines 8 and 9, the lines 12 and 13 define a randomly-initialized
weight matrix {\tt W\_2}, and a zero-initialized bias vector {\tt b\_2}, as
model parameters of the output layer.
Then, the line 14 defines the operation that
multiplies the hidden layer output with
the weight {\tt W\_2}, adds the bias vector {\tt b\_2}, and finally applies the
{\tt softmax} activation function.


%As shown in the lines 5 to 14, developers must explicitly define
%the model components and oprations between them in TensorFlow 1.x model.

%Since the placeholder variables only specify the vector size of the input images
%and the labels; they will be replaced with actual values of the training
%data during the training process. 

%To define a neural network model in TensorFlow 1.x, 
%developers must explicitly define the model structure
%and the operations between the layers. 
%The developers also must manually start the training loop by explicitly
%accessing the TensorFlow runtime with dedicated APIs.

%The lines 8 to 10 defines the hidden layer.
%The line 8 uses the {\tt random\_uniform} API to create 
%a random weight matrix of size 784 by 100, 
%and the line 9 uses the {\tt zero} API to create a zero-vector of size 100.
%Then, the lines 8 and 9 wrap the weight matrix and the bias vector with
%{\tt Variable} API.
%The {\tt Variable} API creates a TensorFlow variable that can be later modified
%during the runtime. It is usually used to define a model 
%parameter, whose value is changed during the training process.
%Thus, the lines 8 and 9 create the weight and bias parameters for the
%hidden layer which have correct sizes and are modifiable during the 
%training process.
%The line 10 manually defines the operations of the hidden layer. 
%Note that the line does not actually compute the operations,
%but only define the operations that will be computed during the training.

After constructing the neural network, the code defines the training algorithm
on the lines 16 and 17.
The line 16 defines the categorical cross entropy loss function that
quantifies the degree of difference between the output data of {\tt layer\_2}
and the answer {\tt y}.
The line 17, then, creates an object of the {\tt AdamOptimizer} that is an
implementation of the Adam gradient descent algorithm~\cite{kingma2014adam}.
The {\tt minimize} method of the object generates a training operation that
updates model parameters using the Adam gradient descent alogrithm to minimize
the loss calculated by the loss function.

%The lines 16 and 17 defines the loss function and the training operation. 
%The line 16 defines the loss function between the model output {\tt layer\_2} 
%and the answer label {\tt y} with categorial cross entropy function.
%The line 17 defines the training operation for the model.
%The line first calls the {\tt AdamOptimizer} constructor
%function to create an optimizer object.
%The optimizer objects in TensorFlow are abstractions of the gradient
%descent algorithms.
%For instance, the {\tt AdamOptimizer} abstracts a Adam gradient descent
%algorithm. % todo: cite Adam g.d.
%Then the {\tt minimize} method defines the training operation that updates the
%TensorFlow variables via the gradient descent on the first argument.
%Thus, the line 17 defines the operation of a single training step,
%which optimizes the model parameters via gradient descent to the loss gradient.

The lines 19 to 22 train the constructed neural network on a session.
The line 19 creates a {\tt Session} object that provides an encapsulated
environment storing model parameters.
%that provides the {\tt run} method to execute the network's operations in an
%encapsulated environment.
%The line 20 calls the {\tt run} method with a global variable initializer
%to initialize model parameters of the network.
The line 20 calls the {\tt run} method of the object with a global variable
initializer to initialize all the model paramters of the network in the
encapsulated environment.
%The {\tt Session} object provides the {\tt run} method that can invoke
%computation of TensorFlow operations.
%Before the training computation starts,
%the TensorFlow variables in the model and the optimizer should be initialized.
%Note that the optimizer object implicitly introduces the variables which are 
%used for the internal computation.
%After the variables are initialized, the line 21 uses the {\tt for} loop
%to iterate over the dataset and get training input data. 
Then, the lines 21 and 22 actually train the network in a loop that executes
the trainig operation {\tt train\_op} with the input data {\tt x} and {\tt y}
by iterating over the ten thousand input dataset obtained from {\tt
dataset.take(10000)}.

%The number of training data is specified by the {\tt take} API of the
%dataset object.
%Finally, the line 22 calls the {\tt run} method to
%invoke computation of the training operation {\tt train\_op}.


\begin{figure}[ht!]
\lstinputlisting[language=Python]
{tensorflow2_ex.py}
  \caption{TensorFlow 2.x model example}
\label{fig:back:tf2}
\end{figure}

% new
Figure~\ref{fig:back:tf2} is a code example of TensorFlow 2.x model.
The developers using TensorFlow 2.x define the model structure and the training
loop. 

The lines 5 to 11 first define the model structure.
First, the lines 5 to 8 use the Keras library APIs to define the neural
network model with a hidden layer and an output layer.
Keras~\cite{keras} is a layer-based deep learning model library included in 
TensorFlow 2.x. The lines 6 and 7 defines the dense layers with the {\tt Dense} 
API. The first argument specifies the size of output vector.
Given the output vector size, the {\tt Dense} API will automatically create
the weight matrix and the bias vector parameters according to the size of 
the previous layer. The second argument of the {\tt Dense} API 
specifies the activation function of the layer,
where the name of the activation function is given as a string. 
The layers are then composed into a linear sequential model
by the {\tt Sequential} API in the line 5.
Compared to TensorFlow 1.x models, developers do not have to explicitly
define the computation between the layers in TensorFlow 2.x.

The lines 10 and 11 define the loss of the model output and
the optimizer object for the model training.
The line 10 uses the {\tt CategoricalCrossentropy} API to 
define the loss function between the model output and the answer label.
The line 11 uses the {\tt Adam} API to define the optimizer object.
This optimizer object is equal to the {\tt AdamOptimizer} object in the 
TensorFlow 1.x model.
However, the usage of the optimizer object is different in two TensorFlow
versions.
In TensorFlow 1.x, the optimizer object is used to define the
training operation {\tt train\_op}, which is later used as an argument
for the {\tt Session.run} method to explicitly trigger the training step.
In TensorFlow 2.x, the optimizer object is used to directly call the methods 
that eagerly executes the gradient descent algorithm against the training input. 

The lines 13 to 19 starts the training loop.
The training loop starts with the {\tt for} loop that iterates over the dataset
and takes batches of training images and answer labels.
Similar to the TensorFlow 1.x model,
the line 13 uses the {\tt take} API to specify the number of batches.
The line 14 creates a new {\tt GradientTape} object by {\tt with} statement.
When a {\tt GradientTape} object is created by {\tt with} statement,
it will record the gradients of the operations inside the {\tt with}
statement body.
The lines 15 and 16 define the forward propagation stage in a single
training step.
Note that in TensorFlow 2.x, the model object created with the Keras APIs can be
used like a Python function as in the line 15.
Thus, the line 15 computes the output vector of the model when the
training image is given as an input.
The line 16 computes the cross entropy loss between the model output and
the answer label.
The lines 18 and 19 then optimizes the model parameters by gradient descent. 
The {\tt model.trainable\_variable} method returns the list of 
TensorFlow variables of the model parameters,
and the {\tt tape.gradient} method returns the list of
gradients for model parameters automatically computed against the
loss value.
The line 19 finally calls the {\tt apply\_gradients} method of the optimizer
object to optimize the model parameters according to gradient descent algorithm.


\subsection{Horovod Distributed Training Library}

Horovod~\cite{sergeev2018horovod} is a Python library for distributed training 
of TensorFlow models. The library adopts model-parallel approach of distributed 
training. 
In model-parallel distributed training, multiple instances of the DL model
are created, and each GPU takes responsibility of computing a single model
instance computation. % <- todo: rephrase?
Note that this means that the number of the model instances 
is equal to the number of GPUs.
In training process, each model instance gets a training batch and computes the
loss value. Then the gradients are averaged across the GPUs to synchronize the
model states. The gradient descent algorithm updates the model parameters by
the averaged gradients.
By the model-parallel distributed training, developers can take advantage of
parallelism to train their DL model in shorter time.

To explain how to rewrite single-GPU TensorFlow models into the distributed 
models with Horovod library, we provide the code examples of ditributing 
the TensorFlow 1.x model in the figure~\ref{fig:back:tf1} and the TensorFlow
2.x model in the figure~\ref{fig:back:tf2}.
We focus on explaining the difference between the original model codes in
the figures~\ref{fig:back:tf1}, \ref{fig:back:tf2} and the distributed 
model codes in the following paragraphs.

\begin{figure}[ht!]
 \lstinputlisting[language=Python]
{horovod_tf1_ex.py}
  \caption{Horovod distributed model example}
\label{fig:back:hvd1} 
\end{figure}

%todo: paragraph 나누는 기준 좀 고민
% p1 : figure intro, explain initialization + gpu pinning
Figure~\ref{fig:back:hvd1} is a code example of using Horovod to distribute
the TensorFlow 1.x model in~\ref{fig:back:tf1}.
The line 4 first initializes the Horovod configuration.
The lines 6 to 8 create the same number of processes with GPUs in the system,
and pins each GPU per process. 
The GPU pinning is a process where each GPU in the system is assigned to a 
single training process.
This is done by refering to the local rank of the process 
by the {\tt local\_rank} API.
This ensures the model training workload is equally distributed over
the GPUs.

% p2: explain optimizer
The line 23 defines the training operation, 
which is a distributed version of the gradient descent algorithm.
Compared to the line 17 in the figure~\ref{fig:back:tf1},
the line 23 makes two changes to the optimizer object.
First, the learning rate argument is multiplied by {\tt hvd.size()}.
According to the Horovod library document,
the learning rate of the distributed optimizer 
should be scaled by the number of GPUs for efficient distributed training.
To acheive this, the line 23 calls the {\tt hvd.size()} method to
get the total number of processes that Horovod has created,
which is equal to the number of GPUs.
Second, the optimizer object is wrapped with the Horovod library API,
{\tt DistributedOptimizer}.
The {\tt DistributedOptimizer} API converts the single GPU-based optimizer 
object to distributed optimizer, which averages the loss gradients across the
training processes before applying the gradient descent.
Thus, wrapping the optimizer with {\tt DistributedOptimizer} API is necessary
for correct distributed training of the model.

% p3: explain variable broadcast
The line 27 broadcasts the model and optimizer variables across the training
processes.
The variable broadcasting is a process where the TensorFlow 
variable values in different training processes are synchronized into
the same value.
According to the Horovod library document, the variable broadcasting
should occur exactly once during the training process,
right after the variables are initialized.
In figure \ref{fig:back:hvd1}, the variable broadcasting is done in the line 27,
which is right after the line 26 that initializes the TensorFlow variables,
and right before the line 28 that starts the training loop.
Note that the {\tt broadcast\_global\_variables} API broadcasts every
TensorFlow variables including the model variables and the optimizer
variables.

% p4 : dataset length scale
Finally, the line 28 starts the training loop.
According to the Horovod library document, the number of training data
batches can be scaled down by the number of GPUs as the same number of
the batches are simultaneously trained in one training step.
Thus, the line 29 divides the number of batches taken from the dataset
by the number of the GPUs, {\tt hvd.size()}.


\begin{figure}[ht!]
 \lstinputlisting[language=Python]
{horovod_ex.py}
  \caption{Horovod distributed model example}
\label{fig:back:hvd2} 
\end{figure}

% todo: detailed expl of gpu pinning also for 1.x?
% todo : detailed xpl of "memory growth" for gpu pininng
% p1: init & gpu pinning
Figure~\ref{fig:back:hvd2} is a code example of using Horovod to distribute
the TensorFlow 2.x model in~\ref{fig:back:tf2}.
The lines 4 and 5 initializes the Horovod configuration and a boolean variable
{\tt hvd\_broadcast\_done}.
The lines 7 to 11 are the GPU pinning code for TensorFlow 2.x. 
The line 7 first gets the list of GPUs in the system.
The line 11 pin the GPU to the process, using the local rank of the process.

% p2 : optimizer lr
The line 19 defines the optimizer object.
As in the optimizer of figure \ref{fig:back:hvd1}, 
the learning rate argument of the optimizer object 
should be scaled by {\tt hvd.size()} for efficient distributed training. 

% p3 : dataset length
The line 21 starts the training loop with the {\tt for} statement.
As mentioned before, the number of training data batches can be
scaled down by the number of GPUs.
To implement this, the {\tt for} loop in the line 21
divides the number of batches taken from the dataset by {\tt hvd.size()}.

% p4 : tape
Inside the training loop,
the line 26 wraps the original {\tt GradientTape} object with the Horovod API  
{\tt DistributedGradientTape}.
The {\tt DistributedGradientTape} API is similar
to the {\tt DistributedOptimizer} API in distributed TensorFlow 1.x model; 
the loss gradients will be averaged across the training processes 
before the gradient descent.
Thus, wrapping the {\tt GradientTape} object with the 
{\tt DistributedGradientTape} API is necessary for correct distributed training
of the model.

% p5. broadcast
Finally, the lines 31 to 34 broadcasts the model and optimizer variables
across the training processes.
As mentioned before, the variable broadcasting should occur exactly once,
right after the variables are initialized.
To ensure the variable broadcasting occur only during the first training step,
the line 31 uses the global boolean variable {\tt hvd\_broadcast\_done},
then changes the value in the line 34 after the broadcasting is done.
Also note that the broadcasting code is placed right after the line 29,
where the optimizer method {\tt apply\_gradient} is finished.
In TensorFlow 2.x, the variables are not explicitly initialized,
but rather implicitly initialized during the first computation involving the
variable.
To ensure that the variable broadcasting happen after the optimizer
variables are initialized, the variable broadcasting codes are placed
after the first gradient descent step is finished, thus the optimizer
variables are already initialized.

\section{Overview}
 
\begin{figure}[ht!]
  \centering
  \includegraphics[width=\textwidth]{overview_diagram.pdf}
  \caption{Overall structure of the 
  automated transformation for distributed training}
  \label{sysarch}
\end{figure}

% main topic sentence for each paragraph

% 우리는 코드 변환을 기반을 기존의 모델을 자동으로 분산화하는 기법을 제안한다.
\noindent
This paper proposes an automated code transformation method that rewrites
TensorFlow DL models to the distributed versions with Horovod.
As discussed in the previous section, distributed training with the Horovod
library requires model engineers to understand the Horovod library and rewrite
model code manually.
To alleviate this burden, our proposed method utilizes static analysis and code
transformation techniques to rewrite TensorFlow DL model code automatically
based on our formal transformation rules.
% to simplify the process of distributing DL models.

%We propose the code transformation-based approach for automated distributed
%training of TensorFlow DL models.
%As explained in the previous section, distributing TensorFlow DL models with
%Horovod library requires adding and editing the original models.
%Currently, the developers must fully understand the model codes and manually
%rewrite them.
%This also includes understanding the Horovod library usage by reading the
%library documentation and code exmaples.
%To ease the burden of the rewriting process, we utilize the code transformation
%technique to automatically rewrite the input TensorFlow model code with Horovod
%library.

Figure \ref{sysarch} illustrates the overview of our automated code
transformation approach for distributed training of DL models with Horovod.
Our approach first parses a given model into Abstract Syntax Trees
(ASTs) to analyze and modify the model code mechanically.
In order to define the code transformation rules for the distributed training
of TensorFlow DL models, we manually inspected the Horovod library
documentation and code examples. 
Through this analysis, we found that different transformation rules are
necessary for TensorFlow models depending on the specific TensorFlow APIs used
in the models.
Thus, we defined four \textit{training API patterns} that represent common code
patterns of TensorFlow APIs that appear in TensorFlow DL models.
The \cha~analyzes the ASTs and extracts the class
inheritance information relations between TensorFlow built-in and
user-defined classes.
Using the inheritance information, the \tapi~identifies the training API
pattern of the input model.  Then, the \atran~selects the appropriate
transformation rules based on the identified training API pattern and applies
the rules to the model's ASTs. 
The modified ASTs are then finally converted back into a TensorFlow DL
model, and the model can now train on multiple GPUs with the support of
the Horovod library.

%The modified ASTs are finally converted into Python codes to return
%the distriubted model as an output.

% 기존의 모델에 적용하는 코드 변환 규칙을 정의하기 위해 horovod 라이브러리 문서와
% 코드 예제를 검토했다. 그 결과로 TF 모델을 4가지로 분류하고 각 분류에 맞는
% 변환 규칙을 정의할 수 있었다.

%To define the code transformation rule for distributed training of 
%TensorFlow DL models,
%we manually inspected the Horovod library documentation and code examples.
%In this end, we identified four categories of TensorFlow DL models.
%We define four \textit{training API patterns}, which are common code patterns of 
%TensorFlow APIs appearing in each category of the TensorFlow DL models.
%To categorize the TensorFlow DL models into one of the four patterns, 
%we implemented the \textit{training API pattern identifier}. 
%In this process, we identified that the training API pattern identifier 
%must know the class inheritance relationship between TensorFlow library 
%classes and user-defined classes.
%To solve this problem, we also implemented the \textit{class hierarchy analyzer}
%to retrieve the class inheritance information of the TensorFlow DL model.



% 우리 기법은 입력으로 주어진 TF 모델에서 먼저 API 패턴을 인식하고
% 해당 패턴에 맞는 코드 변환 규칙을 적용하는 방식으로 작동한다.
% 이를 위해 설계한 우리 기법의 overview는 피규어와 같다...
%Figure \ref{sysarch} illustrates the overview of our approach.
%Given the single-GPU model as an input, our approach parses the model codes
%into ASTs in order to mechanically analyze and modify them.
%We first analyze the class hierarchy of the input TensorFlow model
%to produce the class inheritance information.
%Then the training API pattern identifier uses the inheritance information to
%identify the training API pattern of the input model.
%Finally, the AST transformer selects the correct transformation rule
%according to the training API pattern then apply the rule to the model ASTs.
%The modified ASTs are finally converted into Python codes to return
%the distriubted model as an output.

The subsequent sections provide detailed explanations of each component of our
proposed approach. 
Section \ref{sec:cha} describes the necessity of the class hierarchy analysis
for our approach.
Section \ref{sec:pattern} explains the concept of training API patterns and the
implementation of the \tapi.
Finally, Section \ref{sec:trans} provides a detailed description of the code
transformation process for each identified training API pattern, including the
formalization of the corresponding transformation rules.

%\section{Python Abstract Syntax}\label{sec:pysyn}

To build the Python parser and ASTs for our transformation approach,
we define the abstract syntax of the Python programming language.
We manually examined the grammar specification in 
Python Language Reference \cite{pythonref} to define the formal grammar.
The figure \ref{fig:parse:abssyntax} 
illustrates the simplified version of the Python abstract syntax.
The three major components of the Python syntax are modules, statements, and
expressions.

The top-level component of Python is a module.
The module represents a single Python file, which includes multiple
class definitions and function definitions.
In the abstract syntax, the module is defined to be a statement,
which will be a sequence of multiple statements in most cases.

% todo: rephrase?
% stmt는 등의 프로그램 구성품과 그 행동을 정의한다.
Statements define the program components and their actions. 
In Python, statements are used in two major cases.
First, Python statements define the sequence of actions that changes the
program state. Assignment statements create variables or update the
value stored in the variable. If-else, for, and while statements change
the control flow of the program.
Second, Python statements define the classes and functions.
The {\tt class} statement defines a new class with a class name and the
class methods. 
The {\tt def} statement defines a new function with a function name, the 
argument, and the body statements.

Expressions are parts of the code that evaluate to values.
For instance, list and tuple expressions evlauate to the new list and tuple
objects. Expressions also include operator expressions, which evaluates to
the resulting value of the operation on the operands.
Expressions are used as part of statements, such as assign statements or
control flow statements.

\begin{figure}[!ht]
\begin{tabular}{lrll}
  \nmodule & := & \nstmt  \\
  \nstmt & ::= & \kdef ~ \nid ~ \sparen{\nargs} ~ \kcolon ~ \mul{\nstmt} & \desc{FunDef} \\ 
  & $|$ & \kclass ~ \nid ~ \sparen{\mul{\nexpr} \mul{\nkeyword}} ~ \kcolon ~ \mul{\nstmt} & \desc{ClassDef} \\
  & $|$ & \nexpr ~ \oassign \nexpr & \desc{Assign} \\
  & $|$ & \kfor ~ \nexpr ~ \kin ~ \nexpr ~ \kcolon ~ \mul{\nstmt} & \desc{ForLoop} \\
  & $|$ & \kwhile ~ \sparen{\nexpr} ~ \kcolon ~ \mul{\nstmt} & \desc{WhileLoop} \\
  & $|$ & \kif ~ \sparen{\nexpr} ~ \kcolon ~ \mul{\nstmt} ~ \op{(\kelse ~ \kcolon ~ \mul{\nstmt})}& \desc{If} \\
  & $|$ & \kwith ~ \mul{\nwithitem} ~ \kcolon ~ \mul{\nstmt} & \desc{With} \\
  & $|$ & \kimport ~ \mul{\nalias} & \desc{Import} \\
  & $|$ & \kfrom ~ \nint ~ \op{\nid} \kimport ~ \mul{\nalias} & \desc{ImportFrom} \\
  & $|$ & \nexpr & \desc{ExprStmt} \\
  & $|$ & \nstmt ~ {\tt \textbackslash n} ~ \nstmt  & \desc{Sequence} \\

  \nexpr & ::= & \nexpr ~ \nboolop ~ \nexpr & \desc{BoolOp} \\
  & $|$ & \nexpr ~ \nbinop ~ \nexpr & \desc{BinaryOp} \\ 
  & $|$ & \nunop ~ \nexpr & \desc{UnaryOp} \\ 
  & $|$ & \lparen{\mul{\nexpr}} & \desc{List} \\ 
  & $|$ & \sparen{\mul{\nexpr}} & \desc{Tuple} \\ 
  & $|$ & \nexpr ~ \sparen{\mul{\nexpr} \mul{\nkeyword}} & \desc{Call} \\
  & $|$ & \nconstant & \desc{Constant} \\
  & $|$ & \nexpr {\tt .}\nid& \desc{Attribute} \\
  & $|$ & \nid & \desc{Name} \\

  \nboolop & ::= & \oand ~ $|$ ~ \oor & \desc{BoolOperator} \\
  \nbinop & ::= & \oand ~ $|$ ~ \osub ~ $|$ ~ \omul & \desc{BinOperator} \\
  \nunop& ::= & \kinvert ~ $|$ ~ \knot ~ $|$ ~ \oadd ~ $|$ ~ \osub & \desc{UnOperator} \\
  \nargs & ::= & \mul{(\narg ~ \op{(\oassign ~ \nexpr)})}, ~ \mul{(\narg ~ \op{(\oassign ~ \nexpr)})}, ~ \op{\narg}, ~ \mul{(\narg ~ \op{(\oassign ~ \nexpr)})}, ~ \op{\narg} & \desc{Arguments}\\
  \narg & ::= & \nid ~ \op{\nexpr}~\op{\nstr} & \desc{Argument} \\
  \nkeyword & ::= & \op{\nid} \oassign \nexpr & \desc{Keyword} \\ 
  \nalias & ::= & \nid ~\mul{(.\nid)} \op{(\kas ~ \nid)} & \desc{Alias} \\
  \nwithitem & ::= & \nexpr ~ \op{(\kas ~ \nexpr)} & \desc{WithItem}\\

  \nconstant & ::= & \knone & \desc{NoneLiteral} \\
  & $|$ & \nint & \desc{IntLiteral} \\
  & $|$ & \nfloat & \desc{FloatLiteral} \\
  & $|$ & \nstr & \desc{StringLiteral} \\
  & $|$ & \nbool & \desc{BooleanLiteral} \\
  & $|$ & \sparen{\mul{\nconstant}} & \desc{TupleLiteral} \\
  \nid & $\in$ & \did \\
  \nstr & $\in$ & \dstr \\
  \nbool & $\in$ & \{{\tt True}, {\tt False}\}\\
  \nint & $\in$ & $\mathbb{Z}$ \\
  \nfloat & $\in$ & $\mathbb{R}$ \\
\end{tabular}
  \caption{Python abstract syntax}
  \label{fig:parse:abssyntax}
\end{figure}

% todo : delete pagebrak
\pagebreak

\section{Class hierarchy analysis}\label{sec:cha}

This section explains the necessity of class hierarchy analysis for correct
transformation of the TensorFlow models.
In TensorFlow models, the user-defined classes can inherit 
TensorFlow library classes to extend the methods and behaviours of them 
without modifying the library codes. 
Figure \ref{fig:cha:tfex}(a) shows a code example of user-defined class
inheriting the TensorFlow library class.
In line 4, the newly defined class {\tt ResNet} inherits the TensorFlow
library class {\tt keras.Model}.
The {\tt keras.Model} class represents a DL model
and provides training-related methods such as {\tt compile} or {\tt fit}.
By inheriting the {\tt keras.Model} class, the {\tt ResNet} class
can also utilize the training-related methods while defining the new model
structure.
Line 8 creates a {\tt ResNet} class instance {\tt model}.
Then the line 9 calls the {\tt fit} method which is inherited from the
{\tt keras.Model} class.
By calling the {\tt fit} method, the model instance can be automatically
trained without manually repeating the training steps.

\begin{figure}[!ht]
  \centering
  \begin{subfigure}[t]{0.35\textwidth}
    \begin{lstlisting}[language=Python]
from tensorflow import keras

# `ResNet` inherits `keras.Model`
class ResNet(keras.Model): 
    def __init__(self, block_list):
        ...

model = ResNet([2,2,2])
model.fit(x_train, y_train)\end{lstlisting}
    \caption{Single-GPU-based DL model example using an user-defined class}
  \end{subfigure}
  \hspace{3mm}
  \begin{subfigure}[t]{0.6\textwidth}
    \begin{lstlisting}[language=Python]
from tensorflow import keras
import horovod.tensorflow.keras as hvd

class ResNet(keras.Model):
    def __init__(self, block_list):
        ...

model = ResNet([2,2,2])

model.fit(
    x_train, 
    y_train,
    callbacks=[hvd.callbacks.BroadcastGlobalVariablesCallback(0)])\end{lstlisting}
    \caption{Distributed model example using an user-defined class}
  \end{subfigure}

  \caption{Code example of distributing a model using an user-defined class}
  \label{fig:cha:tfex}
\end{figure}

To correctly recognize the TensorFlow APIs in the TensorFlow model codes, 
we must recognize the inheritance relationship between the user-defined classes 
and TensorFlow library classes.
Figure \ref{fig:cha:tfex}(b) is an example of modifying the code in 
\ref{fig:cha:tfex}(a) into the distributed model.
The line 9 of \ref{fig:cha:tfex}(a) is modified as the lines 10 to 13 of 
\ref{fig:cha:tfex}(b). The modification adds a keyword argument {\tt callbacks}
to the {\tt fit} method call. 
To correclyt modify the code, we must first recognize the the {\tt fit} method
call in the line 9 of \ref{fig:cha:tfex}(a) is inherited from the
{\tt keras.Model} class.
Without recognizing the inheritance relation between {\tt ResNet} and
{\tt keras.Model}, we cannot correctly modify the method call in line 9.  

To solve this problem, we employ the class hierarchy analysis as a pre-analysis.
\textit{Class hierarchy analysis} is a static analysis technique that identifies
inheritance relation between the classes in the code.
By class hierarchy analysis on the input DL models,
we can identify which user-defined classes inherit TensorFlow library classes,
and therefore correctly identify method calls related to the training.
In figure \ref{fig:cha:tfex}(a), for instance, 
the class hierarchy analyzer reads line 4 to conclude that the class
{\tt ResNet} inherits the class {\tt keras.Model}.
This information is sent to the later transformation stage,
so the training API pattern identifier and AST transformer can 
recognize the {\tt fit} method in line 9 as the call to the method from
{\tt keras.Model} class.

\section{Training API Pattern Analysis}\label{sec:pattern}

\subsection{TensorFlow ML Training API Patterns}

\begin{figure}[ht!]
\centering
  \begin{subfigure}[b]{0.4\textwidth}
    \begin{lstlisting}[language=Python]
for x, y in train_data.take(training_steps):
    with tf.GradientTape() as tape:
        pred = model(x, is_training=True)
        loss = loss_compute(y, pred)

    trainable_vars = model.trainable_variables
    gradients = tape.gradient(loss, trainable_vars)
    pairs = zip(gradients, trainable_vars)
    optimizer.apply_gradients(pairs) 
    \end{lstlisting}
    \caption{Using low-level training API}
  \end{subfigure}
  \hspace{5mm}
  \begin{subfigure}[b]{0.45\textwidth}
    \begin{lstlisting}[language=Python]
model.compile(
    optimizer = optimizer, 
    loss = loss_compute) 
model.fit(train_data.take(training_steps))
    \end{lstlisting} 
    \caption{Using high-level training API}
  \end{subfigure}

  \caption{TensorFlow model code example using two different API patterns}
  \label{fig:pattern:ex01}
\end{figure}

TensorFlow provides various APIs to define the training process.
The figure \ref{fig:pattern:ex01} illustrates two examples of using 
different APIs for the model training.
Line 2 to 10 in the code \ref{fig:pattern:ex01}(a)  
uses low-level APIs to repeat training steps over the training
data set. Inside the {\tt for} loop, 
the code uses the {\tt GradientTape} instance to record the loss computation,
then uses the {\tt gradient} method and the {\tt apply\_gradient} method to
back-propagate the gradient to the model parameters.
In contrast, lines 1 and 4 in the code \ref{fig:pattern:ex01}(b)
uses high-level APIs to automatically
invoke the model training process by two methods calls, {\tt compile} and
{\tt fit}. The {\tt fit} method repeats the same process with the {\tt for}
loop in the code \ref{fig:pattern:ex01}(a); it takes each training data from the set and
applies the gradient to the model parameter.

While both code train the model in the same way, 
the code structures are significantly different.
Using high-level APIs simplifies the program,
reducing additional codes that compute the prediction loss and apply gradients.
Low-level APIs are verbose, however, developers can fully control
the training process.
Developers can freely choose from different training code styles to
take advantage of each way.

To correctly transform the training codes, 
the tool must apply different transformation rule for different usages
of training APIs.
We implemented an \textit{API pattern analysis} that identifies
the TensorFlow API usage with \textit{training API patterns} 
Each training API pattern is a set of code patterns for AST,
and code patterns are ASTs with
special holes that are later matched with concrete value.
When a code pattern is matched against a code AST,
it succeeds with the holes matched with the AST subnodes
or fails to match the given AST.
The software matches a training API pattern to the target training code,
and if the match succeeds, the corresponding transformation rule
is used to transform the training code.

We manually inspected multiple TensorFlow training codes to identify
common patterns of training-related API usage. 
We defined four training API patterns
that effectively identify kinds of training APIs usage in the target code.
Figure \ref{tab:patterns} briefly explains each training API pattern.
Note that each pattern is assigned with different TensorFlow versions.
The transformation software also supports legacy training codes
that use compatibility modules for TensorFlow version 1.x.

\begin{figure}
  \centering
  \begin{tabular}{|c|c|l|}
    \hline
    TF version & Pattern name & Explanation \\
    \hline
    1.x & Session & Low-level training API using {\tt Session} instance\\
    \hline
    1.x & MonitoredSession & Low-level training API using {\tt MonitoresSession} instance \\
    \hline
    2.x & GradientTape & Low-level training API using {\tt GradientTape} instance\\
    \hline
    2.x & Keras & High-level training API using {\tt fit} method of {\tt keras.models.Model} instance\\
    \hline
  \end{tabular}
  \caption{Training API patterns}
  \label{tab:patterns}
\end{figure}

We now explain each training API pattern with concrete code examples.

\textbf{Session Pattern} 
The Session pattern refers to TensorFlow 1.x training codes that
uses the {\tt Session} class instance to manually repeat the training 
computation in low-level manner.
Figure \ref{fig:sessionpattern} is a training code example of 
Session pattern.
In line 1, an instance of {\tt tf.compat.v1.train.Session} class is 
created with {\tt with} statement.
The {\tt Session} instance is an object providing a link to 
the TensorFlow runtime, equipped with methods related to the computation graph.
As in line 3, the method {\tt run} is called to invoke the
execution of computation on the pre-defined computation graph.
In the training process, {\tt Session.run} method is repeated mutiple times
so that the forward propagation of the training data and
backward propagation of gradient is performed many times.

To identify a Session pattern training code,
the analyzer searches for a {\tt with} statement that creates a
{\tt Session} instance.
If the body statements of {\tt with} statement calls a 
{\tt run} method of the {\tt Session} instance,
the analyzer concludes that the training code belongs to the Session pattern.

\begin{figure}[!ht]
\begin{lstlisting}[language=Python]
with tf.Session() as sess:
    for step in xrange(int(num_epochs * train_size) // BATCH_SIZE):
      sess.run(optimizer, feed_dict=feed_dict)
\end{lstlisting}
\caption{Session pattern code example}
\label{fig:sessionpattern}
\end{figure}

\textbf{MonitoredSession Pattern}
The MonitoredSession pattern refors to TensorFlow 1.x training codes that
uses the {\tt MonitoredSession} class instance to manually repeat
the training computation in low-level manner.
The pattern is same to the {\tt Session} pattern, except that the
{\tt MonitoredSession} instance is used instead of the {\tt Session} instance.
Figure \ref{fig:monsesspattern} is a code example of MonitoredSession pattern.
In line 1, the {\tt MonitoredSession} class instance is created with
the {\tt tf.traing.MonitoredTrainingSession} call inside {\tt with} statement
(Note that the constructor has different name with the class).
Similar to the Session pattern's case, line 3 calls the {\tt run} method of the
{\tt MonitoredSession} instance to repeat the training computation.

To identify a MoniotoredSession pattern training code,
the analyzer searches for a {\tt with} statement that creates a
{\tt MonitoredSession} instance.
If the body statements calls a {\tt run} methods of the {\tt MonitoredSession}
instance, the analyzer concludes that the training code belongs to the
MonitoredSession pattern.

The {\tt tf.compat.v1.train.MonitoredSession} is used to handle initialization,
recovery, and hooks in the training process\cite{monitoredsession}.
Similar to the {\tt Session} instance, the {\tt run} method of
{\tt MonitoredSession} instance is used to invoke the training computation.
In addition, {\tt MonitoredSession} instances can automatically
initialize and invoke methods of {\tt Hook} objects.
The MonitoresSession pattern matches the constructor calls for the
{\tt MonitoresSession} instance in the training code.

\begin{figure}[!ht]
  \begin{lstlisting}[language=Python]
with tf.train.MonitoredTrainingSession(hooks=hooks) as mon_sess:
    while not mon_sess.should_stop():
        mon_sess.run(train_op, feed_dict=feed_dict)
  \end{lstlisting}
  \label{fig:monsesspattern}
  \caption{MonitoredSession pattern code example}
\end{figure}


\textbf{GradientTape Pattern}
In TensorFlow 2.x, the computations are eagerly evaluated.
Unlike in TensorFlow 1.x, developers do not need to explicitly invoke the computations
with {\tt Session} or {\tt MonitoredSession} instances.
Instead, TensorFlow 2.x training codes use {\tt tf.GradientTape} instances
to record the forward pass operation and automatically compute the gradient.
The GradientTape pattern matches the call expressions {\tt with} statements
that construct {\tt GradientTape} instances.

\begin{figure}[!ht]
  \begin{lstlisting}[language=Python]
for x, y in train_data.take(training_steps):
  with tf.GradientTape() as g:
    pred = conv_net(x)
    loss = cross_entropy(pred, y)
    
  trainable_variables = list(weights.values()) + list(biases.values())
  gradients = g.gradient(loss, trainable_variables)
  optimizer.apply_gradients(zip(gradients, trainable_variables))
  \end{lstlisting}
  \caption{GradientTape pattern code example}
\end{figure}

\textbf{Keras Pattern}
Another common pattern to define the training process is to define
the model with {\tt tf.keras} API and automatically train the model
instance with {\tt fit} method.
In this method, the models are subclasses of {\tt tf.keras.models.Model} class
and inherits training-related method.
The Keras pattern matches the call expressions that is invoking the {\tt fit}
method of the subclass instances of {\tt tf.keras.models.Model}
In the pattern matching process, the API pattern analyzer
makes use of the CHG to identify subclass relationships. 

\begin{figure}[!ht]
  \begin{lstlisting}[language=Python]
class ResNet(tf.keras.models.Model):
  # model definition

model = ResNet([2, 2, 2], num_classes)
model.fit(x_train, y_train_ohe, batch_size=batch_size, epochs=epochs,
  \end{lstlisting}
  \caption{Keras pattern code example}
\end{figure}


\pagebreak
\section{Code Transformation}\label{sec:trans}

\subsection{Formalization of Transformation Rules}

In this section, we describe formalization of the transformation rule
for distributed DL traininig with set of code examples and formal notions
related to them.
There are two main points in design of the formalization.
First, we informally understand the code transformation rule as methods to
\textit{select} the parts of codes to modify and \textit{construct} new code by 
reusing parts of the selected code.
To automate the transformation process, we need code transformation in a form 
that is implements above selection and construction.
Second, as mentioned earlier, to correctly transform tranining codes with different
API patterns, we must apply different transformation rules.
The formalization should also describe different transformation rules
for different API patterns.

In this end, we define the formal transformation rule as a set of
function that takes ASTs as input and returns ASTs as output.
We call these functions \textit{transform function}.
We define a set of transform functions for each training API pattern,
defining total of four sets transform functions.
For the selection and construction of code parts,
we use \textit{pattern matching} as an argument of the transform function;
given an AST as input, the transform function will first match the
input against the argument pattern, then only proceed to apply itself
when the pattern matching is successful.

\begin{figure}[ht!]
  \centering
  \begin{subfigure}[t]{0.48\textwidth}
    \begin{lstlisting}[language=Python]
import tensorflow as tf

dataset = ...
model = ...
optim = tf.optimizers.Adam(0.001) 

for (x, y) in dataset.take(10000):
  with tf.GradientTape() as tape:
    pred = model(x)
    loss_value = loss(y, pred) 

  gradients = tape.gradient(loss_value, model.trainable_variables)
  optim.apply_gradient(zip(gradients, model.trainable_variable)\end{lstlisting}
    \caption{Original DL training code}
  \end{subfigure}
  \hspace{5mm}
  \begin{subfigure}[t]{0.48\textwidth}
    \begin{lstlisting}[language=Python]
import tensorflow as tf
import horovod.tensorflow as hvd

dataset = ...
model = ...
optim = tf.optimizers.Adam(0.001 * hvd.size()) 

for (x, y) in dataset.take(10000):
  with tf.GradientTape() as tape:
    pred = model(x)
    loss_value = loss(y, pred) 
  tape = hvd.DistributedGradientTape(tape)

  gradients = tape.gradient(loss_value, model.trainable_variables)
  optim.apply_gradient(zip(gradients, model.trainable_variable)\end{lstlisting}
    \caption{Distributed DL training code}
  \end{subfigure}
  \caption{Code examples}
  \label{fig:trans:ex}
\end{figure}

We describe examples of the transform function by illustrating the transform
functions applied in the code example in figure \ref{fig:trans:ex}.
Figure \ref{fig:trans:ex}(a) shows an example of the input single-GPU code and
figure \ref{fig:trans:ex}(b) shows the coresponding distributed code. 
Some unnecessary details of the codes are removed from the code.
There are three main parts that are modified.
First, a new import statement that imports {\tt horovod.tensorflow} modules
is added in line 2. 
The statement is added right after the original {\tt tensorflow} module
import statement.
Second, in line 6, an argument of the constructor function 
{\tt tf.optimizers.Adam} is modified. The newly changed argument multiplies
{\tt hvd.size()} onto the original argument.
Third, in line 12, a new statement that assigns the variable {\tt tape}
is added. The assignment wraps the original {\tt GradientTape} object with
the Horovod library function, {\tt hvd.DistributedGradientTape}.
Except the above mentioned parts, the original code is not modified.     

\begin{figure}[ht!]
  \centering

  \begin{subfigure}[t]{0.48\textwidth}
    \begin{lstlisting}[language=Python]
import tensorflow as tf
    \end{lstlisting}
    \caption{Original DL training code: only TensorFlow module is imported.}
  \end{subfigure}
  \hspace{5mm}
  \begin{subfigure}[t]{0.48\textwidth}
    \begin{lstlisting}[language=Python]
import tensorflow as tf
import horovod.tensorflow as hvd
    \end{lstlisting}
    \caption{Distributed DL training code: Horovod module import statement is added.}
  \end{subfigure}
  \caption{Code examples}
  \label{fig:trans:ex01}
\end{figure}

Figure \ref{fig:trans:ex01} is the first transformed part of the code.
The transform function takes an import statement as an input,
identifies that the input statement imports the TensorFlow module,
then outputs a list of two statements, which are the original input 
import statement and the new import statement for the Horovod module. 
The transform function should be able to recognize that the input import
statement actually imports the TensorFlow module.
To acheive this, the transform function utilzes the current transform context.
The transform context stores set of important identifiers related to the
training process; while transforming the code in figure \ref{fig:trans:ex01}(a),
for instance, the identifier {\tt tf} for the TensorFlow
module is stored after line 1, and the identifier {\tt optim}
for the {\tt Optimizer} instance is stored after line 5.
Transform function can compare the transfrom context before the input statement
and just after the input statement to check whether a specific module
is imported in the input statement.

\begin{figure}[ht!]
  \centering
  \noindent
  \begin{tabular}{l}
   \typdesc{\fkstmt& : & \dstmt ~ $\rightarrow$ ~ \dmodenv ~ $\rightarrow$ ~ (\dstmt ~ list ~ $\times$ ~ \dmodenv)}\\
   \tstmt{\kimport ~ \mul{\nalias}}{\smodenv} = \\
    \inden \ktlet ~ \smodenvsubs{1} ~ \kteq ~ \taalias{\mul{\nalias}}{\smodenv} \ktin \\
    \inden \ktif ~ \smodenvsubs{1} ~ \envsub ~ \smodenv ~ \kteq ~ [\tflow $\mapsto$ \nid]\\ 
    \inden\ktthen \\
    \inden\hspace{1em} ([\kimport ~ \mul{\nalias}, \\
    \inden\hspace{1em} \kimport ~ {\tt horovod.tensorflow} \kas ~ {\tt hvd}, \smodenvsubs{1})\\
    \inden \ktelse~([\kimport ~ \mul{\nalias}], \smodenvsubs{1})
  \end{tabular}\\\vpar
  \caption{Transform function: adding Horovod module import statement}
  \label{fig:trans:fn01}
\end{figure}

Figure \ref{fig:trans:fn01} describes the transform function that applies the
transform in figure \ref{fig:trans:ex01}. 
The function specifies input argument with pattern that matches any import
statement; thus the transformation only targets import statements.
As explained earlier, the second and third lines
compare $\smodenv$, the transform context
before the input statement, with $\smodenvsubs{1}$, the transform context
just after the input statement, to check if the input statement
imports the TensorFlow module.
Note that the transform function return the new transform context
$\smodenvsubs{1}$ together with the statement list; this allows
transform functions to propagate the transform context from a call
to the next call.

\begin{figure}[ht!]
  \centering
  \begin{subfigure}[t]{0.48\textwidth}
    \begin{lstlisting}[language=Python]
optim = tf.optimizers.Adam(0.001)\end{lstlisting}
    \caption{Original DL training code}
  \end{subfigure}
  \hspace{5mm}
  \begin{subfigure}[t]{0.48\textwidth}
    \begin{lstlisting}[language=Python]
optim = tf.optimizers.Adam(0.001 * hvd.size())\end{lstlisting}
    \caption{Distributed DL training code}
  \end{subfigure}
  \caption{Code examples}
  \label{fig:trans:ex02}
\end{figure}

Figure \ref{fig:trans:ex02} illustrates the second part of the transformed code.
The assign statement in \ref{fig:trans:ex02}(a) is transformed to
another assign statement in \ref{fig:trans:ex02}(b), where the argument for the
constructor {\tt tf.optimizers.Adam} is multiplied by the expression
{\tt hvd.size()}. 

\subsection{Description of Transform functions}

As mentioned earlier, training codes of different API patterns
require different transformation rules to correctly transform them.
This section explains formal transform functions for each API patterns.

\subsubsection{Transform function for Session pattern}

To correctly transform Session pattern trainig codes into corresponding
distributed training codes, the {\tt Optimizer} instance must be modified.
Figure \ref{fig:trans:sessiontrans} shows an pair of code examples
illustrating the required transformation for Session pattern case.
In line 1 of figure \ref{fig:trans:sessiontrans}(b),
the {\tt learning\_rate} argument is scaled by {\tt hvd.size()}.
Then the line 2 is added, which wraps the {\tt Optimizer} instance
by designated Horovod API.

\begin{figure}[ht!]
  \begin{subfigure}[t]{0.45\textwidth}
    \begin{lstlisting}[language=Python]
optimizer = tf.train.MomentumOptimizer(learning_rate = 0.01)

with tf.Session() as sess:
    for step in range(num_epochs): 
        sess.run(optimizer, feed_dict)
    \end{lstlisting}
    \caption{Original training code of Session pattern}
  \end{subfigure}
  \hspace{5mm}
  \begin{subfigure}[t]{0.45\textwidth}
    \begin{lstlisting}[language=Python]
optimizer = tf.train.MomentumOptimizer(learning_rate = 0.01 * hvd.size())
optimizer = hvd.DistributedOptimizer(optimizer)

with tf.Session() as sess:
    for step in range(num_epochs): 
        sess.run(optimizer, feed_dict)
    \end{lstlisting}
    \caption{Distributed training code of Session pattern}
  \end{subfigure}
  \caption{Code example of Session pattern training code}
  \label{fig:trans:sessiontrans}
\end{figure}

The above transformation is formalized as shown in the figure 
\ref{fig:trans:sessrule}. The transform function stated in the figure
matches statements that assigns a evaluation result of function call expression
on the right-hand side.
If an assign statement is matched, the pattern guard checks if the
callee function is the {\tt Optimizer} instance constructor function.
Note that the pattern guard checks this by subclass relation between
the function expression and {\tt tensorflow.keras.optimizers.Optimizer}.
This way, the pattern guard can accept class constructors for {\tt Optimizer}
instances, including both TensorFlow library and user-defined classes.

The transform function proceeds to return a sequence of two statements;
the first one is modified version of the input assign statement,
and the second one is additional assign statement for wrapping the optimizer instance.
The first statement modifies the original {\tt learning\_rate} argument
by constructing a new expression, {\tt \nexprsubs{2i} * hvd.size()}.
The transform function can easily construct new ASTs by using
the pattern variables so that the part of the input statement can be conveniently
referred. This is also true for building the second output statement, which
reused the {\tt \nidsubs{r}} to construct a new assign statment.

\begin{figure}[h]
\noindent
\begin{longtable}{l}
  \tstmt{\nidsubs{r} \oassign \nexprsubs{1} \sparen{\nexprsubs{11} ... \nexprsubs{1n} ~ \op{(\nidsubs{1} \oassign)} \nexprsubs{21} ... \op{(\nidsubs{k} \oassign)} \nexprsubs{2k}} }{\smodenv} = \\
\inden \ktif ~ \nexprsubs{1} \ktsubty ~ {\tt tensorflow.keras.optimizers.Optimizer} ~ \ktthen\\
  \inden\inden \ktif ~ \nidsubs{i} ~ \kteq ~ {\tt learning\_rate} ~ \ktwhen ~ 1 $\leq$ i $\leq$ k ~ \ktthen\\
  \inden\inden\inden ([\nidsubs{r} \oassign \nexprsubs{1} \sparen{\nexprsubs{11} ... \nexprsubs{1n} ~ \op{(\nidsubs{1} \oassign)} \nexprsubs{21} ... \nidsubs{i} \oassign \nexprsubs{2i} {\tt * hvd.size()}\\
  \inden\inden\inden\inden ... \op{(\nidsubs{k} \oassign)} \nexprsubs{2k}} \optypcomm \\
  \inden\inden\inden {\tt \nidsubs{r} = hvd.DistributedOptimizer(\nidsubs{r})}],
  \smodenv[\optmizer $\mapsto$ \nidsubs{r}])\\
\end{longtable}
  \caption{Session pattern transform function: Optimizer learning rate scaling and wrapping}
  \label{fig:trans:sessrule}
\end{figure}

\subsubsection{Transform function for MonitoredSession pattern}

The MonitoredSession pattern codes use {\tt MonitoredSession} instances
to automatically repeat training steps.
By default, a training step includes feeding an training data to the network,
computing gradients, and optimize the network parameters.
In addition the default actions, users can add \textit{hooks},
a custom action that automatically executes for each training step.
For example, a {\tt CheckpointSaverHook} instance can be added to
save the model status into a file at each step.

In distributed training with Horovod, one of the multiple training processes
must broadcast the global variables states to every other processes so that
they can synchronize the initial model states.
To correctly transform MonitoredSession pattern training codes into
distributed training codes, the {\tt MonitoredSession} instances should have
variable broadcast hooks.
As in figure \ref{fig:trans:monsesstrans}, for example,
the correct transformation appends {\tt BroadcastGlobalVariablesHook} to the
{\tt hooks} arguments.
The {\tt hooks} keyword argument gets the hook list for the training steps;
the transformed code adds the Horovod library hook that broadcases the
global variable at the first training step.

\begin{figure}[ht!]
  \centering
  \begin{subfigure}[t]{0.45\textwidth}
    \begin{lstlisting}[language=Python]
with tf.train.MonitoredTrainingSession(hooks=hooks) as mon_sess:
    while not mon_sess.should_stop():
        mon_sess.run()
    \end{lstlisting}
    \caption{Original training code of MonitoredSession pattern}
  \end{subfigure}
  \hspace{5mm}
  \begin{subfigure}[t]{0.45\textwidth}
    \begin{lstlisting}[language=Python]
with tf.train.MonitoredTrainingSession(hooks=hooks.append(hvd.BroadcastGlobalVariablesHook(0)) as mon_sess:
    while not mon_sess.should_stop():
        mon_sess.run()
    \end{lstlisting}
    \caption{Distributed training code of MonitoredSession pattern}
  \end{subfigure}
  \caption{Code example of MonitoredSession pattern training code}
  \label{fig:trans:monsesstrans}
\end{figure}

We formalize the transformation for MonitoredSession pattern as illustrated in
figure \ref{fig:trans:monsessrule}. 
The transform function matches \textit{with item}, an expression that is 
assigned by the {\tt with} statement, which is a function call expression of
the {\tt MonitoredSession} class constructor.
Note that the pattern guard refers to the environment variable to
get the identifier of TensorFlow 1.x compatability module.
This ensures the transform function to correctly identify the
{\tt MonitoredSession} class constructor from the TensorFlow module.   

\begin{figure}[ht!]
 \noindent
  \begin{tabular}{l}
    \twithitem{\nexprsubs{1} \sparen{\nexprsubs{11} ... \nexprsubs{1n} ~ 
                \op{(\nidsubs{1} \oassign)} \nexprsubs{21} ... 
                \op{(\nidsubs{k} \oassign)} \nexprsubs{2k}}}{\smodenv} = \\
    % monitored session
    \inden \ktif ~ \smodenv(\tflowc) ~ \kteq ~ \nidsubs{t} ~ \ktand \\
    \inden\inden \nexprsubs{1} ~ \kteq ~ 
                {\tt \nidsubs{t}.train.MonitoredTrainingSession} ~ \ktthen \\
    \inden\inden \ktif ~ \nidsubs{i} ~ \kteq ~ {\tt hooks} ~ \ktwhen ~ 1 $\leq$ i $\leq$ k ~ \ktthen\\
    \inden\inden\inden(\nexprsubs{1} \sparen{\nexprsubs{11} ... 
              \nexprsubs{1n} ~ \op{(\nidsubs{1} \oassign)} \nexprsubs{21} \\
    \inden\inden\inden\inden ... \nidsubs{i} \oassign {\tt \nexprsubs{2i}.append(hvd.BroadcastGlobalVariablesHook(0))} \\
    \inden\inden\inden\inden ... \op{(\nidsubs{k} \oassign)} \nexprsubs{2k} }, \\
    \inden\inden\inden\inden \smodenv[$\msess$ $\mapsto$ \kas]) \\
  \end{tabular}\\\vpar
\caption{MonitoredSession pattern transform function: Modifying {\tt StopAtStepHook} instance}
  \label{fig:trans:monsessrule}
\end{figure}

\subsubsection{Transform function for GradientTape pattern}

To correctly transform GradientTape pattern training codes,
the {\tt GradientTape} instance should be modified for distributed training.
Figure \ref{fig:trans:gtapetrans} illustrates the transformation of
GradientTape pattern code to the distributed code.
As shown in the figure \ref{fig:trans:gtapetrans}(b), line 8,
the {\tt GradientTape} instance {\tt tape}
is wrapped by {\tt DistributedGradientTape}.
In addition to modifying the {\tt GradientTape} instance,
line 13 to 17 adds variable broadcasting codes in the distributed training
code. Note that the variable {\tt hvd\_broadcast\_done} is initialized as 
{\tt False} at line 3, 
then set to {\tt True} at line 17 where the variable broadcasting is done. 
We use the boolean variable {\tt hvd\_broadcast\_done} in GrdientTape pattern
distributed training code to ensure that variable broadcasting occurs exactly
once.

\begin{figure}[ht!]
  \centering
  \begin{subfigure}[t]{0.45\textwidth}
    \begin{lstlisting}[language=Python]
import tensorflow as tf

with tf.GradientTape() as tape:
    probs = model(images)
    loss_value = loss(labels, probs)

grads = tape.gradient(loss_value, model.trainable_variables)
opt.apply_gradients(zip(grads, model.trainable_variables))
    \end{lstlisting}
    \caption{Original training code of GradientTape pattern}
  \end{subfigure}
  \hspace{5mm}
  \begin{subfigure}[t]{0.45\textwidth}
    \begin{lstlisting}[language=Python]
import tensorflow as tf
import horovod.tensorflow as hvd
hvd_broadcast_done = False

with tf.GradientTape() as tape:
    probs = model(images)
    loss_value = loss(labels, probs)
tape = hvd.DistributedGradientTape(tape)
grads = tape.gradient(loss_value, model.trainable_variables)
id_new = zip(grads, model.trainable_variables)
opt.apply_gradients(id_new)

global hvd_broadcast_done
if not hvd_broadcast_done:
    hvd.broadcast_variables([x[1] for x in id_new], root_rank=0, )
    hvd.broadcast_variables(opt.variables(), root_rank=0, )
    hvd_broadcast_done = True
    \end{lstlisting}
    \caption{Distributed training code of GradientTape pattern}
  \end{subfigure}
  \caption{Code example of GradientTape pattern training code}
  \label{fig:trans:gtapetrans}
\end{figure}

Figure \ref{fig:trans:gtaperule} illustrates a GradientTape pattern
transform function.
The function is responsible of wrapping the
{\tt GradientTape} instance with the {\tt DistributedGradientTape} API.
As illustrated in the figure, the function matches {\tt with} statements.
The pattern \textit{with\_item} matches the pair of identifier and expression 
that {\tt with} statement newly creates. 
For example, the pattern matches a pair of the identifier {\tt tf} and the 
expression {\tt tf.GradientTape()} when line 3 of figure 
\ref{fig:trans:gtapetrans}(a) is given.
In the second to fourth line, the function utilizes the environment variable  
to check if the input {\tt with} statement creates the a new {\tt GradientTape}
instance.
If the statement does create a new {\tt GradientTape} instance,
then the function appends a new statement that reassigns the identifier
wrapped with the {\tt DistributedGraidentTape} API.

\begin{figure}[ht!]
\noindent
\begin{tabular}{l}
  \tstmt{\kwith ~ \nwithitem ~ \kcolon ~ \mul{\nstmt}}{\smodenv} = \\
  \inden \ktlet ~ \nwithitem$'$, \smodenvsubs{1} \kteq ~ \twwithitem{\nwithitem}{\smodenv} \ktin \\
  \inden \ktlet ~ \nstmt$'$, \smodenvsubs{2} \kteq ~ \tsstmt{\mul{\nstmt}}{\smodenvsubs{1}} \ktin \\
  \inden \ktif ~ \smodenvsubs{1} \envsub ~ \smodenv ~ \kteq ~ [\gtape $\mapsto$ \nid] ~ \ktthen\\
  \inden\inden ([\optypcomm ~ \kwith ~ \nwithitem$'$ ~ \kcolon ~ \mul{\nstmt}$'$, \\
  \inden\inden \nid ~ \oassign {\tt hvd.DistributedGradientTape(\nid)}], \smodenvsubs{2})\\
  \inden \ktelse ~ ([\kwith ~ \nwithitem$'$ ~ \kcolon ~ \mul{\nstmt}$'$], \smodenvsubs{2})
\end{tabular}
  \caption{GradientTape pattern transform function: wrapping {\tt GradientTape} instance}
  \label{fig:trans:gtaperule}
\end{figure}

Figure \ref{fig:trans:gtaperule2} illustrates another GradientTape pattern
transform function which is responsible of appending the variable broadcasting
code. The function matches statements that assigns the value of function call
results. Then the pattern guard checks if the called function is the
{\tt apply\_gradients} method of the {\tt Optimizer} instance.
Here, the guard retreives the name of the {\tt Optimizer} instance from the
environment variable. 
The transform function finally returns a fixed sequence of statements that
performs the variable broadcast on the model and optimizer state variables.
Note that we utilize the list comprehension to extract the second element
of each pair in the list expression \nidsubs{z} as line 15 in figure
\ref{fig:trans:gtapetrans}(b).

\begin{figure}[ht!]
\noindent
\begin{tabular}{l}
  \tstmt{\nidsubs{r} \oassign \nexprsubs{1} \sparen{\nexprsubs{11} ... \nexprsubs{1n} ~ \op{(\nidsubs{1} \oassign)} \nexprsubs{21} ... \op{(\nidsubs{k} \oassign)} \nexprsubs{2k}} }{\smodenv} = \\
  \inden \ktif  ~ \smodenv(\optmizer) ~ \kteq ~ \nidsubs{t} ~ \ktand ~ \nexprsubs{1} ~ \kteq ~ {\tt \nidsubs{t}.apply\_gradients} ~ \ktthen\\
  \inden\inden \ktlet ~ \nidsubs{z} ~ \kteq ~ \newid ~ \ktin \\
  \inden\inden ([\nidsubs{z} ~ \oassign ~ \nexprsubs{11},\\
  \inden\inden \nidsubs{r} \oassign \nexprsubs{1} \sparen{\nidsubs{z} \nexprsubs{12} ... \nexprsubs{1n} ~ \op{(\nidsubs{1} \oassign)} \nexprsubs{21} ... \op{(\nidsubs{k} \oassign)} \nexprsubs{2k}} ,\\
  \inden\inden {\tt if not hvd\_broadcast\_done:} \\ 
  \inden\inden\inden [ {\tt hvd.broadcast\_variables([x[1] for x in \nidsubs{z}], root\_rank=0)}, \\
  \inden\inden\inden {\tt hvd.broadcast\_variables(\nidsubs{t}.variables(), root\_rank=0)}, \\
  \inden\inden\inden {\tt hvd\_broadcast\_done = True} ]\\
  \inden\inden ], \smodenv) \\


\end{tabular}
  \caption{GradientTape pattern transform function: adding variable broadcasting code}
  \label{fig:trans:gtaperule2}
\end{figure}

\subsubsection{Transform function for Keras pattern}

In the training codes matching the Keras pattern,
the {\tt Model} instances are created and used to invoke the
training process. The transform function first stores the {\tt Model} instance
variable name in the environment as described in figure \ref{fig:trans:ker}.
The AST pattern guard tests that the given function expression is a subclass
of {\tt tensorflow.keras.Model}, which corresponds to the class constructors
of subclass instances of {\tt tensorflow.keras.Model} class.

\begin{figure}[ht!]
\begin{longtable}{l}
  \tstmt{\nidsubs{r} \oassign \nexprsubs{1} \sparen{\nexprsubs{11} ... \nexprsubs{1n} ~ \op{(\nidsubs{1} \oassign)} \nexprsubs{21} ... \op{(\nidsubs{k} \oassign)} \nexprsubs{2k}} \optypcomm}{\smodenv} = \\
  \inden \ktelif ~ \nexprsubs{1} ~ \ktsubty ~ {\tt tensorflow.keras.Model} ~ \ktthen\\
  \inden\inden ([\nidsubs{r} \oassign \nexprsubs{1} \sparen{\nexprsubs{11} ... \nexprsubs{1n} ~ \op{(\nidsubs{2} \oassign)} \nexprsubs{21} ... \op{(\nidsubs{k} \oassign)} \nexprsubs{2k}}], \smodenv[\tmodel \nidsubs{r}])\\
\end{longtable}
  \caption{Keras pattern transform function: storing the {\tt Model} instance}
  \label{fig:trans:ker}
\end{figure}

The training process is invoked with {\tt fit} method of the {\tt Model} instance.
For the training codes matching the Keras pattern, the variable broadcast should
be done by adding the {\tt hvd.callbacks.BroadcastGlobalVariablesCallback(root\_rank=0)}
callback to the {\tt fit} method argument.
The transform function utilizes the environment parameter to identify the {\tt fit}
method call and insert the callback object to the argument.
Figure \ref{fig:trans:ker2} describes the transform function that
adds the callback to the {\tt fit} function call expression.

\begin{figure}[ht!]
\begin{longtable}{l}
  \tstmt{\nexprsubs{1} \sparen{\nexprsubs{11} ... \nexprsubs{1n} ~ \op{(\nidsubs{1} \oassign)} \nexprsubs{21} ... \op{(\nidsubs{k} \oassign)} \nexprsubs{2k}}}{\smodenv} = \\
  \inden \ktelif ~ \nidsubs{m} ~ \kteq ~ \smodenv(\tmodel) ~ \ktand ~ 
          \nexprsubs{1} ~ \kteq ~ {\tt \nidsubs{m}.fit} ~ \ktthen \\
  \inden\inden \ktif ~ \nidsubs{i} ~ \kteq ~ {\tt callbacks} ~ \ktwhen ~ 2 $\leq$ i $\leq$ k ~ \ktthen \\
  \inden\inden\inden ([{\tt callbacks = [hvd.callbacks.BroadcastGlobalVariablesCallback(root\_rank=0)]},\\
  \inden\inden\inden ~ {\tt if hvd.rank() == 0: callbacks.append(\nexprsubs{2i})}, \\
  \inden\inden\inden ~ {\tt \nexprsubs{1} (optim ... \nexprsubs{1n}}
                              \op{(\nidsubs{1} \oassign)} \nexprsubs{21} ... 
                              {\tt callbacks \oassign callbacks} ... 
                              \op{(\nidsubs{k} \oassign)} \nexprsubs{2k}{\tt )}], \smodenv) \\
  \inden\inden \ktelse \\
  \inden\inden\inden ([{\tt callbacks = [hvd.callbacks.BroadcastGlobalVariablesCallback(root\_rank=0)]},\\
  \inden\inden\inden ~ {\tt \nexprsubs{1} (optim ... \nexprsubs{1n}}
                              \op{(\nidsubs{1} \oassign)} \nexprsubs{21} ... 
                              \op{(\nidsubs{k} \oassign)} \nexprsubs{2k}...
                              {\tt callbacks \oassign callbacks} {\tt )}], \smodenv) \\ 
\end{longtable}
  \caption{Keras pattern transform function: adding the broadcast callback}
  \label{fig:trans:ker2}
\end{figure}

The Keras pattern transform function also modifies the {\tt Optimizer} instance.
First, the learning rate parameter of the instance should be scaled by
{\tt hvd.size()}. Second, the instance should be wrapped with
{\tt hvd.Distributed Optimizer} function, similar to the GradientTape pattern's case.
Figure \ref{fig:trans:ker3} describes the transform function modifying
the {\tt Optimizer} instance.

\begin{figure}[ht!]
\begin{longtable}{l}
  \tstmt{\nidsubs{r} \oassign \nexprsubs{1} \sparen{\nexprsubs{11} ... \nexprsubs{1n} ~ \op{(\nidsubs{1} \oassign)} \nexprsubs{21} ... \op{(\nidsubs{k} \oassign)} \nexprsubs{2k}} \optypcomm}{\smodenv} = \\
  \inden \ktelif ~ \nexprsubs{1} \ktsubty ~ {\tt tensorflow.keras.optimizers.Optimizer} ~ \ktthen\\
  \inden\inden \ktif ~ \nidsubs{i} ~ \kteq ~ {\tt learning\_rate} ~ \ktwhen ~ 1 $\leq$ i $\leq$ k ~ \ktthen\\
  \inden\inden\inden ([\nidsubs{r} \oassign \nexprsubs{1} \sparen{\nexprsubs{11} ... \nexprsubs{1n} ~ \op{(\nidsubs{1} \oassign)} \nexprsubs{21} ... \nidsubs{i} \oassign \nexprsubs{2i} {\tt * hvd.size()}\\
  \inden\inden\inden\inden ... \op{(\nidsubs{k} \oassign)} \nexprsubs{2k}} \optypcomm \\
  \inden\inden\inden {\tt \nidsubs{r} = hvd.DistributedOptimizer(\nidsubs{r})}],
  \smodenv[\optmizer $\mapsto$ \nidsubs{r}])\\
  \inden\inden \ktelse \\
  \inden\inden\inden ([\nidsubs{r} \oassign \nexprsubs{1} \sparen{\nexprsubs{11}
  {\tt * hvd.size()} ... \nexprsubs{1n} ~ \op{(\nidsubs{1} \oassign)}
\nexprsubs{21} ... \op{(\nidsubs{k} \oassign)} \nexprsubs{2k}} \optypcomm, \\
  \inden\inden\inden{\tt \nidsubs{r} = hvd.DistributedOptimizer(\nidsubs{r})}],
  \smodenv[\optmizer $\mapsto$ \nidsubs{r}])\\
\end{longtable}
  \caption{Keras pattern transform function: modifying the {\tt Optimizer} instance}
  \label{fig:trans:ker3}
\end{figure}

\section{Evaluation}\label{sec:eval}
We should devise our evalution metric showing usefulness of our approach.
Also, we should collect target TensorFlow ML models used in our evaluation.

\section{Related Work}\label{sec:related}
We need to survey some related work and describe them here.

\section{Conclusion}\label{sec:conclusion}

We presented the automated code transformation for distributed training.  
By manually inspecting the documentations and code example,
we defined \textit{training API patterns} for categorizing TensorFlow
training codes by their API usage, and constructed \textit{transformation rules}
to distribute the trainig codes of each tranining patterns.
We implement the transformation in a form of software,
which includes class hierarchy analysis and pattern analysis to recognize
the correct transformation rule for the given input model
and automatically apply the transformation to produce
the corresponding distributed model as an output.
We evaluated the correctness of the transformation tool against
16 open-source DL models, which all but one transformations are
successful. Evaluating the training performance of the
transformed model showed us that none or only minimal amounts of
hyperparameter tuning is required for distributed training speedup. 
We believe that our transformation tool frees the users from
heavy burden of rewriting the model code, and allows them to
swiftly move from single-GPU-based training to distributed training.
In future works, we aim to search for methods that can fully automate
the deployments of DL models on distributed systems, including
automated hyperparameter tunings suited for the distributed system.


\section*{Acknowledgements}

This work was supported by Institute for Information \& communications
Technology Planning \& Evaluation(IITP) grant funded by the Korea government
(MSIT)(No.RS-2022-II221200, Convergence security core talent training
business(Chungnam National University)).


\clearpage

\bibliographystyle{elsarticle-num}
\bibliography{ref}

%\bibliography{sn-bibliography}% common bib file

% TODO(all): Add your biography here.
%\begin{biography}{\includegraphics[width=66pt,height=86pt,draft]{empty}}{\textbf{Author Name.} This is sample author biography text this is sample author biography text this is sample author biography text this is sample author biography text this is sample author biography text this is sample author biography text this is sample author biography text this is sample author biography text this is sample author biography text this is sample author biography text this is sample author biography text this is sample author biography text this is sample author biography text this is sample author biography text this is sample author biography text this is sample author biography text this is sample author biography text this is sample author biography text this is sample author biography text this is sample author biography text this is sample author biography text.}
%\end{biography}

\end{document}
