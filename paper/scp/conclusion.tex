\section{Conclusion}\label{sec:conclusion}
We propose the automated approach to transform TensorFlow DL models written in
Python to models training on multiple GPUs.
\begin{inred}
We categorized TensorFlow DL models by four patterns of training API usage,
and devised a static analysis technique that identifies the training pattern
of the given model code.
Then we defined code transformation rules for each training API pattern
that parallelize the training process via Horovod library APIs.
To this end, we implemented a code transformation tool
that takes a single-GPU-based DL model training code,
identifies its training pattern, and applies the corresponding source code
transformation to output a distributed training code for the same DL model.

We conducted two experiments to evaluate our proposed approach.
First, we show that our approach correctly transforms 15 out of 16
open-source TensorFlow LD models.
Second, we show that our approach efficiently distributes the training process
so that the transformed models train about 2.28 times faster than the original models.
By an additional experiment,
we claim that the hyperparameters of the transformed model can be tuned to
obtain better training speeds.
We argue that our tool reduces developers' burden in rewriting models
in accordance with the documentation of distributed training libraries to
parallelize training.

While our work has enabled automatically distributing the training of TensorFlow
DL models, the limitation of our approach can be solved in further studies.
Future work could improve our code transformation rules so that 
we can correctly transform a larger set of TensorFlow DL models and applications.
Many state-of-the-art models and real-world DL applications
depend on external training libraries or non-Python scripts to train the model.
To correctly transform them into distributed models,
our approach can be extended to identify API patterns of various training libraries
and transform them into distributed versions.
We also expect that our code transformation approach can be extended to
support other deep learning libraries such as PyTorch~\cite{pytorch2019},
and to other distributed training frameworks such as DeepSpeed~\cite{deepspeed}.
Finally, to correctly transform DL training codes in real-world applications,
future work can develop a static analysis technique that separates 
and transforms only the parts related to DL model training from the application code. 

Moreover, the training performance of automatically transformed models
could be further improved by automatically finding optimal hyperparameters.
Several previous works~\cite{pmlr-v33-yogatama14}\cite{autohyper-rl}\cite{autotune}
already proposed techniques to automatically tune hyperparameters
of specific models.
By incorporating the techniques from these previous studies into the pipeline of 
our approach, we can further improve the distributed training performance.
By addressing these future directions,
we believe that a broader range of DL developer community can benefit from
our code transformation approach.
\end{inred}


% We identified four common training patterns for TensorFlow DL models and formal
% code transformation rules for each pattern to parallelize training via Horovod
% APIs.
% Also, we developed a code transformation tool that takes a TensorFlow DL model,
% identifies its training pattern via static analysis techniques, and rewrites
% it for distributed training by applying transformation rules of the identified
% training pattern.
% The evaluation showed that our approach is practical in correctly transforming
% 15 out of 16 open-source TensorFlow DL models.
% %to the same with their handcrafted distributed training versions.
% We also showed that our approach is effective in that the transformed models
% train about 2.28 times faster than the original models.
% We believe that our tool reduces model engineers' burdens in rewriting models
% in accordance with the documentation of distributed training libraries to
% parallelize training.

%Our approach classifies TensorFlow DL models into four common training patterns
%we defined, based on their usage of the TensorFlow APIs.
%Then, our approach rewrites models with Horovod APIs by applying transformation
%rules we formally defined for each training pattern.

%By manually inspecting the Horovod document and code examples,
%we defined \textit{training API patterns} for categorizing TensorFlow
%training codes by their API usage, and constructed \textit{transformation rules}
%to distribute the trainig codes of each tranining patterns.

%We implement the transformation in a form of software,
%which includes class hierarchy analysis and pattern analysis to recognize
%the correct transformation rule for the given input model
%and automatically apply the transformation to produce
%the corresponding distributed model as an output.

%We evaluated the correctness of the transformation tool against
%16 open-source DL models, which all but one transformations are
%successful. 

%Evaluating the training performance of the
%transformed model showed us that none or only minimal amounts of
%hyperparameter tuning is required for distributed training speedup. 

%We believe that our transformation tool frees the users from heavy burden of
%rewriting the model code, and allows them to swiftly move from single-GPU-based
%training to distributed training.

%In future works, we aim to search for methods that can fully automate
%the deployments of DL models on distributed systems, including
%automated hyperparameter tunings suited for the distributed system.

\section{Data Availability}
The datasets generated and/or analyzed during the current study are available in 
\href{https://github.com/kaist-plrg/python-analyzer}{https://github.com/kaist-plrg/python-analyzer}.
