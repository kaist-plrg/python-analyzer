\section{Python Abstract Syntax}\label{sec:pysyn}

To build the Python parser and ASTs for our transformation approach,
we define the abstract syntax of the Python programming language.
We manually examined the grammar specification in 
Python Language Reference \cite{pythonref} to define the formal grammar.
The figure \ref{fig:parse:abssyntax} 
illustrates the simplified version of the Python abstract syntax.
The three major components of the Python syntax are modules, statements, and
expressions.

The top-level component of Python is a module.
The module represents a single Python file, which includes multiple
class definitions and function definitions.
In the abstract syntax, the module is defined to be a statement,
which will be a sequence of multiple statements in most cases.

% todo: rephrase?
% stmt는 등의 프로그램 구성품과 그 행동을 정의한다.
Statements define the program components and their actions. 
In Python, statements are used in two major cases.
First, Python statements define the sequence of actions that changes the
program state. Assignment statements create variables or update the
value stored in the variable. If-else, for, and while statements change
the control flow of the program.
Second, Python statements define the classes and functions.
The {\tt class} statement defines a new class with a class name and the
class methods. 
The {\tt def} statement defines a new function with a function name, the 
argument, and the body statements.

Expressions are parts of the code that evaluate to values.
For instance, list and tuple expressions evlauate to the new list and tuple
objects. Expressions also include operator expressions, which evaluates to
the resulting value of the operation on the operands.
Expressions are used as part of statements, such as assign statements or
control flow statements.

\begin{figure}[!ht]
\begin{tabular}{lrll}
  \nmodule & := & \nstmt  \\
  \nstmt & ::= & \kdef ~ \nid ~ \sparen{\nargs} ~ \kcolon ~ \mul{\nstmt} & \desc{FunDef} \\ 
  & $|$ & \kclass ~ \nid ~ \sparen{\mul{\nexpr} \mul{\nkeyword}} ~ \kcolon ~ \mul{\nstmt} & \desc{ClassDef} \\
  & $|$ & \nexpr ~ \oassign \nexpr & \desc{Assign} \\
  & $|$ & \kfor ~ \nexpr ~ \kin ~ \nexpr ~ \kcolon ~ \mul{\nstmt} & \desc{ForLoop} \\
  & $|$ & \kwhile ~ \sparen{\nexpr} ~ \kcolon ~ \mul{\nstmt} & \desc{WhileLoop} \\
  & $|$ & \kif ~ \sparen{\nexpr} ~ \kcolon ~ \mul{\nstmt} ~ \op{(\kelse ~ \kcolon ~ \mul{\nstmt})}& \desc{If} \\
  & $|$ & \kwith ~ \mul{\nwithitem} ~ \kcolon ~ \mul{\nstmt} & \desc{With} \\
  & $|$ & \kimport ~ \mul{\nalias} & \desc{Import} \\
  & $|$ & \kfrom ~ \nint ~ \op{\nid} \kimport ~ \mul{\nalias} & \desc{ImportFrom} \\
  & $|$ & \nexpr & \desc{ExprStmt} \\
  & $|$ & \nstmt ~ {\tt \textbackslash n} ~ \nstmt  & \desc{Sequence} \\

  \nexpr & ::= & \nexpr ~ \nboolop ~ \nexpr & \desc{BoolOp} \\
  & $|$ & \nexpr ~ \nbinop ~ \nexpr & \desc{BinaryOp} \\ 
  & $|$ & \nunop ~ \nexpr & \desc{UnaryOp} \\ 
  & $|$ & \lparen{\mul{\nexpr}} & \desc{List} \\ 
  & $|$ & \sparen{\mul{\nexpr}} & \desc{Tuple} \\ 
  & $|$ & \nexpr ~ \sparen{\mul{\nexpr} \mul{\nkeyword}} & \desc{Call} \\
  & $|$ & \nconstant & \desc{Constant} \\
  & $|$ & \nexpr {\tt .}\nid& \desc{Attribute} \\
  & $|$ & \nid & \desc{Name} \\

  \nboolop & ::= & \oand ~ $|$ ~ \oor & \desc{BoolOperator} \\
  \nbinop & ::= & \oand ~ $|$ ~ \osub ~ $|$ ~ \omul & \desc{BinOperator} \\
  \nunop& ::= & \kinvert ~ $|$ ~ \knot ~ $|$ ~ \oadd ~ $|$ ~ \osub & \desc{UnOperator} \\
  \nargs & ::= & \mul{(\narg ~ \op{(\oassign ~ \nexpr)})}, ~ \mul{(\narg ~ \op{(\oassign ~ \nexpr)})}, ~ \op{\narg}, ~ \mul{(\narg ~ \op{(\oassign ~ \nexpr)})}, ~ \op{\narg} & \desc{Arguments}\\
  \narg & ::= & \nid ~ \op{\nexpr}~\op{\nstr} & \desc{Argument} \\
  \nkeyword & ::= & \op{\nid} \oassign \nexpr & \desc{Keyword} \\ 
  \nalias & ::= & \nid ~\mul{(.\nid)} \op{(\kas ~ \nid)} & \desc{Alias} \\
  \nwithitem & ::= & \nexpr ~ \op{(\kas ~ \nexpr)} & \desc{WithItem}\\

  \nconstant & ::= & \knone & \desc{NoneLiteral} \\
  & $|$ & \nint & \desc{IntLiteral} \\
  & $|$ & \nfloat & \desc{FloatLiteral} \\
  & $|$ & \nstr & \desc{StringLiteral} \\
  & $|$ & \nbool & \desc{BooleanLiteral} \\
  & $|$ & \sparen{\mul{\nconstant}} & \desc{TupleLiteral} \\
  \nid & $\in$ & \did \\
  \nstr & $\in$ & \dstr \\
  \nbool & $\in$ & \{{\tt True}, {\tt False}\}\\
  \nint & $\in$ & $\mathbb{Z}$ \\
  \nfloat & $\in$ & $\mathbb{R}$ \\
\end{tabular}
  \caption{Python abstract syntax}
  \label{fig:parse:abssyntax}
\end{figure}

% todo : delete pagebrak
\pagebreak
