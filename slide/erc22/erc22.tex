\documentclass{beamer}

\title{Automated Code Transformation for Distributed Training of TensorFlow ML Models}
\author{Yusung Sim\inst{1}, Wonho Shin\inst{1}, Sungho Lee\inst{2}}
\institute{
  \inst{1}%
  School of Computing, KAIST
  \and
  \inst{2}%
  Department of Computer Science and Engineering, Chungnam National University
}
\date{2022}

\begin{document}
% 0. Title
\frame{\titlepage}


% 1. Introduction
\begin{frame}
  \frametitle{Distributed ML Training}
  Distributed ML Training scales well \& reduces training time.
\end{frame}

\begin{frame}
  \frametitle{Challenge: Manual Rewriting}
  Rewriting single-GPU based model into distributed model
  \begin{itemize}
    \item Manual rewriting is tedious and time-consuming
    \item Developer must understand distributed training API
    \item Actual code changes are mostly syntactic and simple
  \end{itemize}
\end{frame}

\begin{frame}
  \frametitle{Automatic code transformation}
  Propose automatic code transformation for distributed ML training
  \begin{itemize}
    \item Define formal transformation rule
    \item Implement the rule as a software
  \end{itemize}
\end{frame}

% 2. Background
\begin{frame}
  \frametitle{Horovod: Distributed Training Framework}
  Horovod is a distributed ML training framework developed by Uber, 2017.
\end{frame}

% 3. Method
\begin{frame}
  \frametitle{Python Abstract Syntax}
  We manually investigated the Python Language Reference,
  defined the Python abstract syntax.
\end{frame}

\begin{frame}
  \frametitle{Training API Patterns for TensorFlow ML Model}
  TensorFlow provides multiple training APIs.
  Different training API usage requires different transformation rule 
\end{frame}

\begin{frame}
  \frametitle{Formal Transformation Rule for Distributed ML Model}
  Code transformation is formally defined as AST $\rightarrow$ AST
  
\end{frame}


\end{document}
